\documentclass[12pt,a4paper]{article}
\usepackage[utf8]{inputenc}
\usepackage[T2A]{fontenc}
\usepackage[russian]{babel}
\usepackage{amsmath,amssymb,amsfonts}
\usepackage{geometry}
\geometry{margin=2cm}

\title{\S 4. Реальные газы}
\author{Конспект формул}
\date{}

\begin{document}
\maketitle

%═══════════════════════════════════════════════════════════════════
\section{Краткая теория}
%═══════════════════════════════════════════════════════════════════

\subsection{Потенциал межмолекулярного взаимодействия}

Потенциальная энергия взаимодействия $\Phi(r)$ между двумя частицами газа:
\begin{itemize}
    \item При малых $r$: $\Phi \to +\infty$ (отталкивание, ``непроницаемость'' атомов)
    \item При больших $r$: $\Phi \to 0^{-}$ (притяжение)
    \item Минимум: $\Phi_0 = |\min\{\Phi(r)\}| \sim kT_K$
\end{itemize}
где $T_K$ --- критическая температура вещества.

\subsection{Уравнение Ван-дер-Ваальса}

Учитывает:
\begin{itemize}
    \item Силы притяжения между молекулами $\Rightarrow$ дополнительное давление $\nu^2 a/V^2$
    \item Силы отталкивания $\Rightarrow$ ``собственный объём'' молекул $\nu b$
\end{itemize}

\subsection{Критическая точка}

При $T = T_K$ изотерма имеет точку перегиба (точка $K$):
\begin{itemize}
    \item $T > T_K$: только газообразное состояние
    \item $T < T_K$: возможен фазовый переход жидкость--газ
\end{itemize}

%═══════════════════════════════════════════════════════════════════
\section{Основные формулы}
%═══════════════════════════════════════════════════════════════════

\subsection{Уравнение Ван-дер-Ваальса}

\begin{equation}
    \boxed{\left(p + \frac{\nu^2 a}{V^2}\right)(V - \nu b) = \nu RT}
    \tag{4.1}
\end{equation}

где $\nu = m/\mu$ --- число молей, $a$ и $b$ --- константы Ван-дер-Ваальса (разные для разных газов).

\textbf{Для одного моля} ($\nu = 1$):
\begin{equation}
    \left(p + \frac{a}{V^2}\right)(V - b) = RT
    \tag{4.2}
\end{equation}

или в явном виде для давления:
\begin{equation}
    p = \frac{RT}{V - b} - \frac{a}{V^2}
\end{equation}

%───────────────────────────────────────────────────────────────────
\subsection{Критические параметры}

В критической точке: $\left(\frac{\partial p}{\partial V}\right)_{T_K} = 0$ и $\left(\frac{\partial^2 p}{\partial V^2}\right)_{T_K} = 0$.

\textbf{Выражение констант через критические параметры:}
\begin{equation}
    \boxed{a = \frac{27 R^2 T_K^2}{64 p_K}}, \qquad
    \boxed{b = \frac{RT_K}{8p_K}}
\end{equation}

\textbf{Критический объём:}
\begin{equation}
    V_K = 3\nu b
\end{equation}

\textbf{Число молей через критические параметры:}
\begin{equation}
    \nu = \frac{8 V_K p_K}{3 R T_K}
\end{equation}

%───────────────────────────────────────────────────────────────────
\subsection{Внутренняя энергия газа Ван-дер-Ваальса}

\begin{equation}
    \boxed{U = -\frac{\nu^2 a}{V} + \nu C_V T}
\end{equation}

При $a = 0$ переходит в выражение для идеального газа.

Частная производная:
\begin{equation}
    \left(\frac{\partial U}{\partial V}\right)_T = \frac{\nu^2 a}{V^2}
\end{equation}

%───────────────────────────────────────────────────────────────────
\subsection{Энтропия газа Ван-дер-Ваальса}

\begin{equation}
    \boxed{S = \nu C_V \ln T + \nu R \ln(V - \nu b) + S_0}
\end{equation}

\textbf{Важно:} $S_{\text{Ван-дер-Ваальс}} < S_{\text{идеальный газ}}$ при одинаковых $\nu$, $V$, $T$.

%───────────────────────────────────────────────────────────────────
\subsection{Разность теплоёмкостей}

\begin{equation}
    \boxed{C_p - C_V = R\left[1 - \frac{2a(V-b)^2}{RTV^3}\right]^{-1}}
\end{equation}

Для газа Ван-дер-Ваальса: $C_p - C_V > R$ (больше, чем для идеального газа).

\textbf{Теплоёмкость $C_V$} не зависит от объёма $V$:
\begin{equation}
    \left(\frac{\partial C_V}{\partial V}\right)_T = T\left(\frac{\partial^2 p}{\partial T^2}\right)_V = 0
\end{equation}

%───────────────────────────────────────────────────────────────────
\subsection{Эффект Джоуля--Томсона для газа Ван-дер-Ваальса}

При $b \ll V$ и $a/V^2 \ll p \approx RT/V$:
\begin{equation}
    \boxed{\left(\frac{\partial T}{\partial p}\right)_H = \frac{2a/RT - b}{C_p}}
\end{equation}

\textbf{Температура инверсии:}
\begin{equation}
    \boxed{T^* = \frac{2a}{bR}}
\end{equation}

\begin{itemize}
    \item $T < T^*$: газ \textbf{охлаждается} при расширении (N$_2$, O$_2$, воздух)
    \item $T > T^*$: газ \textbf{нагревается} при расширении (H$_2$, He при комнатной $T$)
\end{itemize}

%═══════════════════════════════════════════════════════════════════
\section{Полезные соотношения из примеров}
%═══════════════════════════════════════════════════════════════════

\subsection{Адиабатическое расширение в пустоту}

Внутренняя энергия сохраняется ($\Delta U = 0$). Изменение температуры:
\begin{equation}
    T_1 - T_2 = \frac{2a(V_2 - V_1)}{5R V_1 V_2}
\end{equation}
(для двухатомного газа, $C_V = 5R/2$)

\subsection{Смешение двух порций газа Ван-дер-Ваальса}

При смешении газов в теплоизолированных сосудах ($\Delta U = 0$):
\begin{equation}
    T_2 = T_1 - \frac{a(V_2 - V_1)^2}{2C_V V_1 V_2 (V_2 + V_1)}
\end{equation}

Давление после установления равновесия:
\begin{equation}
    p = \frac{2RT_2}{V_1 + V_2 - 2b} - \frac{4a}{(V_1 + V_2)^2}
\end{equation}

\subsection{Работа при изобарическом расширении}

Работа газа Ван-дер-Ваальса при изобарическом расширении равна работе при изотермическом расширении между теми же объёмами:
\begin{equation}
    A = p(V_2 - V_1) = RT\ln\frac{V_2 - b}{V_1 - b} + \frac{a}{V_2} - \frac{a}{V_1}
\end{equation}

\subsection{Частные производные для уравнения Ван-дер-Ваальса}

\begin{equation}
    \left(\frac{\partial T}{\partial V}\right)_p = \frac{TRV^3 - 2a(V-b)^2}{RV^3(V-b)}
\end{equation}

\begin{equation}
    \left(\frac{\partial p}{\partial T}\right)_V = \frac{R}{V - b}
\end{equation}

\end{document}
