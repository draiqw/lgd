\documentclass[12pt,a4paper]{article}
\usepackage[utf8]{inputenc}
\usepackage[T2A]{fontenc}
\usepackage[russian]{babel}
\usepackage{amsmath,amssymb,amsfonts}
\usepackage{geometry}
\geometry{margin=2cm}

\title{\S 1. Основные понятия и первое начало термодинамики}
\author{Конспект формул}
\date{}

\begin{document}
\maketitle

%═══════════════════════════════════════════════════════════════════
\section{Краткая теория}
%═══════════════════════════════════════════════════════════════════

\subsection{Основные определения}

\textbf{Термодинамическая система} --- совокупность $N \sim N_A$ хаотически движущихся частиц в ограниченной области пространства.

\textbf{Параметры системы:}
\begin{itemize}
    \item \textbf{Внешние} $a_i$ $(i = 1, 2, \ldots, n)$ --- определяются состоянием тел вне системы
    \item \textbf{Внутренние} $b_j$ --- определяются движением частиц системы
    \item \textbf{Интенсивные} --- не зависят от массы (давление, температура)
    \item \textbf{Экстенсивные} --- пропорциональны массе (объём, энергия)
\end{itemize}

\textbf{Термодинамическое равновесие} --- состояние изолированной системы, из которого она самопроизвольно выйти не может. Характеризуется абсолютной температурой $T$ [K].

\textbf{Равновесный (квазистатический) процесс} --- процесс, при котором система в каждый момент времени находится в состоянии термодинамического равновесия.

\textbf{Теплоёмкость} $\tilde{C} = \delta Q / dT$ --- количество теплоты для изменения температуры на $dT$.

\textbf{Адиабатический процесс} --- процесс без теплообмена с окружающей средой ($\delta Q = 0$).

%═══════════════════════════════════════════════════════════════════
\section{Основные формулы}
%═══════════════════════════════════════════════════════════════════

\subsection{Фундаментальные константы}

\begin{align}
    N_A &= 6{,}02 \cdot 10^{23} \text{ моль}^{-1} && \text{(число Авогадро)} \\[6pt]
    R &= 8{,}31 \text{ Дж/(моль$\cdot$К)} && \text{(универсальная газовая постоянная)} \\[6pt]
    k &= \frac{R}{N_A} = 1{,}38 \cdot 10^{-23} \text{ Дж/К} && \text{(постоянная Больцмана)}
\end{align}

%───────────────────────────────────────────────────────────────────
\subsection{Уравнения состояния}

\textbf{Термическое уравнение состояния (общий вид):}
\begin{equation}
    B_i = B_i(a_1, a_2, \ldots, T)
    \tag{1.1}
\end{equation}
где $B_i$ --- обобщённая термодинамическая сила, сопряжённая параметру $a_i$.

\textbf{Уравнение Клапейрона--Менделеева (идеальный газ):}
\begin{equation}
    \boxed{pV = \frac{m}{\mu}RT = \nu RT}
    \tag{1.2}
\end{equation}
где $\nu = m/\mu$ --- число молей, $\mu$ --- молярная масса.

\textbf{Калорическое уравнение состояния (внутренняя энергия идеального газа):}
\begin{equation}
    \boxed{U = \frac{i}{2}kTN = \frac{i}{2}\frac{m}{\mu}RT}
    \tag{1.3}
\end{equation}
где $i$ --- число степеней свободы молекулы.

%───────────────────────────────────────────────────────────────────
\subsection{Условие равновесного процесса}

\begin{equation}
    \frac{da_i}{dt} \ll \frac{\Delta a_i}{\tau}, \qquad \frac{db_j}{dt} \ll \frac{\Delta b_j}{\tau}
    \tag{1.4}
\end{equation}
где $\tau$ --- время релаксации системы.

%───────────────────────────────────────────────────────────────────
\subsection{Работа и теплота}

\textbf{Элементарная работа (общий случай):}
\begin{equation}
    \delta A = \sum_{i=1}^{n} B_i \, da_i
    \tag{1.5}
\end{equation}

\textbf{Работа газа при изменении объёма:}
\begin{equation}
    \boxed{A = \int_{V_0}^{V_1} p(V) \, dV}
    \tag{1.6}
\end{equation}

\textbf{Теплоёмкость системы:}
\begin{equation}
    \tilde{C} \equiv \frac{\delta Q}{dT}
    \tag{1.7}
\end{equation}

Связь теплоёмкостей: $C = \tilde{C}/\nu$, \quad $c = \tilde{C}/m$, \quad где $\nu = m/\mu$.

%───────────────────────────────────────────────────────────────────
\subsection{Первое начало термодинамики}

\begin{equation}
    \boxed{\delta Q = dU(a_1, \ldots, a_n, T) + \delta A}
    \tag{1.8}
\end{equation}

\textbf{Для кругового процесса (цикла):}
\begin{equation}
    A = Q_1 + Q_2 = Q_1 - |Q_2|
    \tag{1.9}
\end{equation}
где $Q_1$ --- теплота от нагревателя, $Q_2$ --- теплота холодильнику.

\textbf{КПД тепловой машины:}
\begin{equation}
    \boxed{\eta = \frac{A}{Q_1}}
\end{equation}

\textbf{Холодильный коэффициент:}
\begin{equation}
    \eta_x = \frac{|Q_2|}{A}
\end{equation}

%───────────────────────────────────────────────────────────────────
\subsection{Теплоёмкость идеального газа}

\textbf{Теплоёмкость при постоянном объёме:}
\begin{equation}
    \boxed{C_V = \frac{i}{2}R}
\end{equation}

\textbf{Теплоёмкость при постоянном давлении:}
\begin{equation}
    \boxed{C_p = C_V + R = \frac{i+2}{2}R}
\end{equation}

\textbf{Показатель адиабаты:}
\begin{equation}
    \boxed{\gamma = \frac{C_p}{C_V} = \frac{i+2}{i}}
\end{equation}

\textbf{Общая формула теплоёмкости:}
\begin{equation}
    C = C_V + p\frac{dV}{dT}
    \tag{1.10}
\end{equation}

%───────────────────────────────────────────────────────────────────
\subsection{Политропный процесс}

\textbf{Уравнение политропы:}
\begin{equation}
    \boxed{pV^n = \text{const}}
    \tag{1.11}
\end{equation}

\textbf{Показатель политропы:}
\begin{equation}
    n = \frac{C - C_p}{C - C_V}
\end{equation}

\textbf{Теплоёмкость в политропном процессе:}
\begin{equation}
    C(n) = C_V \frac{n - \gamma}{n - 1}
\end{equation}

\textbf{Частные случаи:}
\begin{center}
\begin{tabular}{|c|c|c|c|}
\hline
$n$ & Процесс & Условие & Теплоёмкость \\
\hline
$0$ & Изобарический & $p = \text{const}$ & $C_p$ \\
$1$ & Изотермический & $T = \text{const}$ & $\pm\infty$ \\
$\gamma$ & Адиабатический & $\delta Q = 0$ & $0$ \\
$\infty$ & Изохорический & $V = \text{const}$ & $C_V$ \\
\hline
\end{tabular}
\end{center}

%───────────────────────────────────────────────────────────────────
\subsection{Термические коэффициенты}

\textbf{Коэффициент объёмного расширения:}
\begin{equation}
    \alpha = \frac{1}{V_0}\left(\frac{\partial V}{\partial T}\right)_p
\end{equation}

\textbf{Температурный коэффициент давления:}
\begin{equation}
    \lambda = \frac{1}{p_0}\left(\frac{\partial p}{\partial T}\right)_V
\end{equation}

\textbf{Изотермическая сжимаемость:}
\begin{equation}
    \beta = -\frac{1}{V_0}\left(\frac{\partial V}{\partial p}\right)_T
\end{equation}

\textbf{Для идеального газа при $T_0 = 273$ К:}
\begin{equation}
    \alpha = \lambda = \frac{1}{T_0}, \qquad \beta = \frac{1}{p_0}, \qquad \alpha = p_0 \beta \lambda
\end{equation}

%───────────────────────────────────────────────────────────────────
\subsection{Цикл Карно}

\textbf{КПД цикла Карно (идеальный газ):}
\begin{equation}
    \boxed{\eta = 1 - \frac{T_2}{T_1}}
\end{equation}
где $T_1$ --- температура нагревателя, $T_2$ --- температура холодильника.

\textbf{Теплота на изотерме:}
\begin{equation}
    Q_{12} = \nu RT_1 \ln\frac{V_2}{V_1}
\end{equation}

%───────────────────────────────────────────────────────────────────
\subsection{Разность теплоёмкостей (общий случай)}

Для любого однородного изотропного вещества:
\begin{equation}
    \tilde{C}_p - \tilde{C}_V = \left[\left(\frac{\partial U}{\partial V}\right)_T + p\right]\left(\frac{\partial V}{\partial T}\right)_p
\end{equation}

%═══════════════════════════════════════════════════════════════════
\section{Полезные соотношения из примеров}
%═══════════════════════════════════════════════════════════════════

\textbf{Изменение внутренней энергии идеального газа:}
\begin{equation}
    \Delta U = \frac{m}{\mu} C_V \Delta T = \nu C_V \Delta T
\end{equation}

\textbf{Теплота в изобарическом процессе:}
\begin{equation}
    Q_p = \nu C_p \Delta T
\end{equation}

\textbf{Теплота в изохорическом процессе:}
\begin{equation}
    Q_V = \nu C_V \Delta T = \Delta U
\end{equation}

\textbf{Работа в изобарическом процессе:}
\begin{equation}
    A = p \Delta V = \nu R \Delta T
\end{equation}

\textbf{Работа в изотермическом процессе:}
\begin{equation}
    A = \nu RT \ln\frac{V_2}{V_1}
\end{equation}

\textbf{Связь холодильного коэффициента и КПД (машина Карно):}
\begin{equation}
    \eta_x = \frac{|Q_2|}{A} = \frac{1-\eta}{\eta} = \eta^{-1} - 1
\end{equation}

\end{document}
