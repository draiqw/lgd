\documentclass[12pt,a4paper]{article}
\usepackage[utf8]{inputenc}
\usepackage[T2A]{fontenc}
\usepackage[russian]{babel}
\usepackage{amsmath,amssymb,amsfonts}
\usepackage{geometry}
\geometry{margin=2cm}

\title{\S 2. Второе начало термодинамики. Энтропия}
\author{Конспект формул}
\date{}

\begin{document}
\maketitle

%═══════════════════════════════════════════════════════════════════
\section{Краткая теория}
%═══════════════════════════════════════════════════════════════════

\subsection{Формулировки второго начала термодинамики}

\textbf{Формулировка Томсона (Кельвина):}
Невозможно осуществить циклический процесс, единственным результатом которого было бы превращение в механическую работу количества теплоты, отнятой у какого-нибудь тела, без каких-либо изменений в другом теле.

\textbf{Формулировка Карно:}
Циклическая тепловая машина, работающая при данных температурах нагревателя и холодильника, не может иметь больший КПД, чем машина Карно при тех же температурах.

\subsection{Энтропия}

\textbf{Энтропия} $S$ --- функция состояния термодинамической системы, определяемая через полный дифференциал. Измеряется в Дж/К.

\textbf{Свойства энтропии:}
\begin{itemize}
    \item В изолированной системе: $\Delta S \geq 0$ (равенство для обратимых процессов)
    \item В неизолированной системе: $\Delta S$ может быть любого знака
    \item Все естественные процессы необратимы $\Rightarrow$ энтропия замкнутой системы возрастает
\end{itemize}

%═══════════════════════════════════════════════════════════════════
\section{Основные формулы}
%═══════════════════════════════════════════════════════════════════

\subsection{Дифференциал энтропии}

\textbf{Для равновесного (обратимого) процесса:}
\begin{equation}
    \boxed{dS \equiv \frac{\delta Q}{T} = \frac{dU}{T} + \frac{1}{T}\sum_{i=1}^{n} B_i \, da_i}
    \tag{2.1}
\end{equation}

\textbf{Для неравновесного процесса:}
\begin{equation}
    dS > \frac{\delta Q'}{T} = \frac{dU}{T} + \frac{1}{T}\sum_{i=1}^{n} B_i \, da_i
    \tag{2.2}
\end{equation}

\textbf{Важно:} $T$ --- температура \textit{отдающего} тепло тела!

%───────────────────────────────────────────────────────────────────
\subsection{Изменение энтропии при конечном переходе}

\textbf{Общая формула:}
\begin{equation}
    \boxed{S_2 - S_1 = \int_1^2 \frac{\delta Q}{T} = \int_1^2 T^{-1} dU + \int_1^2 T^{-1} \sum_{i=1}^{n} B_i \, da_i}
    \tag{2.3}
\end{equation}

Интеграл вычисляется вдоль \textit{любой} равновесной траектории между состояниями 1 и 2.

\textbf{Для одного моля идеального газа:}
\begin{equation}
    \boxed{S_2 - S_1 = R \ln\frac{V_2}{V_1} + C_V \ln\frac{T_2}{T_1}}
    \tag{2.4}
\end{equation}

%───────────────────────────────────────────────────────────────────
\subsection{Неравенство Клаузиуса}

\textbf{Для произвольного кругового процесса:}
\begin{equation}
    \boxed{\oint \frac{\delta Q}{T} \leq 0}
    \tag{2.5}
\end{equation}

\begin{itemize}
    \item Знак ``$=$'' --- для обратимых процессов
    \item Знак ``$<$'' --- для необратимых процессов
\end{itemize}

%───────────────────────────────────────────────────────────────────
\subsection{Первое начало через энтропию}

\begin{equation}
    \delta Q = TdS(V,T) = dU(V,T) + pdV = \left(\frac{\partial U}{\partial T}\right)_V dT + \left(\frac{\partial U}{\partial V}\right)_T dV + pdV
    \tag{2.6}
\end{equation}

%───────────────────────────────────────────────────────────────────
\subsection{Соотношение для давления и внутренней энергии}

\begin{equation}
    \boxed{T\left(\frac{\partial p}{\partial T}\right)_V = p + \left(\frac{\partial U}{\partial V}\right)_T}
    \tag{2.7}
\end{equation}

Связывает $p$ и $U(T,V)$ в любой термодинамической системе.

%───────────────────────────────────────────────────────────────────
\subsection{Энтропия идеального газа (явный вид)}

\begin{equation}
    S = \frac{m}{\mu}\left(R \ln V + C_V \ln T + S_0\right)
    \tag{2.8}
\end{equation}

где $S_0$ --- константа интегрирования (зависит от числа частиц).

%───────────────────────────────────────────────────────────────────
\subsection{Закон Стефана--Больцмана}

Для равновесного электромагнитного излучения:
\begin{equation}
    \boxed{u = \sigma T^4}
\end{equation}
где $u = U/V$ --- плотность энергии излучения, $\sigma = 7{,}64 \cdot 10^{-16}$ Дж/(К$^4\cdot$м$^3$) --- постоянная Стефана--Больцмана.

Давление излучения: $p = u/3$.

%═══════════════════════════════════════════════════════════════════
\section{Полезные соотношения из примеров}
%═══════════════════════════════════════════════════════════════════

\subsection{Изменение энтропии в типовых процессах}

\textbf{Изотермическое расширение идеального газа:}
\begin{equation}
    \Delta S = \frac{m}{\mu} R \ln\frac{V_2}{V_1} = \frac{pV}{T} \ln\frac{V_2}{V_1}
\end{equation}

\textbf{Нагревание тела:}
\begin{equation}
    \Delta S = mc \int_{T_1}^{T_2} \frac{dT}{T} = mc \ln\frac{T_2}{T_1}
\end{equation}
где $c$ --- удельная теплоёмкость.

\textbf{Фазовый переход при температуре $T$:}
\begin{equation}
    \Delta S = \frac{Q}{T} = \frac{m\lambda}{T}
\end{equation}
где $\lambda$ --- удельная теплота фазового перехода.

%───────────────────────────────────────────────────────────────────
\subsection{КПД цикла Карно (через энтропию)}

На диаграмме $T$--$S$ цикл Карно --- прямоугольник.

\textbf{Теплота на изотерме 1--2 (нагреватель):}
\begin{equation}
    Q_{12} = T_1(S_2 - S_1)
\end{equation}

\textbf{Теплота на изотерме 3--4 (холодильник):}
\begin{equation}
    Q_{34} = T_2(S_1 - S_2)
\end{equation}

\textbf{Работа за цикл:}
\begin{equation}
    A = (T_1 - T_2)(S_2 - S_1)
\end{equation}

\textbf{КПД:}
\begin{equation}
    \boxed{\eta = \frac{A}{Q_{12}} = \frac{T_1 - T_2}{T_1} = 1 - \frac{T_2}{T_1}}
\end{equation}

\textbf{Теорема Карно:} КПД цикла Карно не зависит от рода рабочего тела, а только от температур нагревателя и холодильника.

%───────────────────────────────────────────────────────────────────
\subsection{Установление теплового равновесия}

Два тела с теплоёмкостями $\tilde{C}_p = \text{const}$ и температурами $T_1$, $T_2$ приходят в контакт.

\textbf{Равновесная температура:}
\begin{equation}
    T_3 = \frac{T_1 + T_2}{2}
\end{equation}

\textbf{Изменение энтропии системы:}
\begin{equation}
    \Delta S = \tilde{C}_p \ln\frac{T_3^2}{T_1 T_2} = \tilde{C}_p \ln\frac{(T_1 + T_2)^2}{4T_1 T_2} > 0
\end{equation}

(т.к. $(T_1 + T_2)/2 \geq \sqrt{T_1 T_2}$ --- неравенство между средним арифметическим и геометрическим)

%───────────────────────────────────────────────────────────────────
\subsection{Диффузия газов}

При смешении двух идеальных газов с одинаковой температурой:
\begin{equation}
    \Delta S = \left(\frac{M_1}{\mu_1} + \frac{M_2}{\mu_2}\right) R \ln 2 > 0
\end{equation}

\textbf{Парадокс Гиббса:} формула даёт $\Delta S > 0$ даже для одинаковых газов, хотя состояние системы не меняется. Разрешается корректным учётом зависимости $S_0$ от числа частиц.

%───────────────────────────────────────────────────────────────────
\subsection{Необратимое адиабатическое расширение в пустоту}

Для $\nu$ молей идеального газа при расширении объёма в 2 раза:
\begin{equation}
    \Delta S_1 = \nu R \ln 2, \qquad \Delta U_1 = 0
\end{equation}

При обратном адиабатическом сжатии:
\begin{equation}
    \Delta S_2 = 0, \qquad T = T_0 \cdot 2^{\gamma - 1}
\end{equation}

Итоговое изменение:
\begin{equation}
    \Delta U = \nu C_V T_0 (2^{\gamma-1} - 1)
\end{equation}

%───────────────────────────────────────────────────────────────────
\subsection{Максимальная работа тепловой машины}

Два тела с теплоёмкостями $\tilde{C}_1$, $\tilde{C}_2$ и начальными температурами $T_{10}$, $T_{20}$.

\textbf{Конечная равновесная температура:}
\begin{equation}
    T = T_{10}^{\tilde{C}_1/(\tilde{C}_1+\tilde{C}_2)} \cdot T_{20}^{\tilde{C}_2/(\tilde{C}_1+\tilde{C}_2)}
\end{equation}

\textbf{Максимальная работа:}
\begin{equation}
    A_{\max} = \tilde{C}_1 T_{10} + \tilde{C}_2 T_{20} - (\tilde{C}_1 + \tilde{C}_2) T
\end{equation}

\end{document}
