\documentclass[12pt,a4paper]{article}
\usepackage[utf8]{inputenc}
\usepackage[T2A]{fontenc}
\usepackage[russian]{babel}
\usepackage{amsmath,amssymb,amsfonts}
\usepackage{geometry}
\geometry{margin=2cm}

\title{\S 3. Термодинамические потенциалы}
\author{Конспект формул}
\date{}

\begin{document}
\maketitle

%═══════════════════════════════════════════════════════════════════
\section{Краткая теория}
%═══════════════════════════════════════════════════════════════════

\subsection{Термодинамические потенциалы}

\textbf{Термодинамические потенциалы} --- функции состояния, позволяющие выразить все макроскопические характеристики системы и анализировать устойчивость равновесия.

\begin{center}
\begin{tabular}{|c|c|c|}
\hline
Потенциал & Определение & Естественные переменные \\
\hline
Внутренняя энергия $U$ & --- & $V$, $S$ \\
Свободная энергия $F$ & $U - TS$ & $V$, $T$ \\
Энтальпия $H$ & $U + pV$ & $p$, $S$ \\
Потенциал Гиббса $G$ & $U - TS + pV$ & $p$, $T$ \\
\hline
\end{tabular}
\end{center}

\subsection{Условия равновесия и устойчивости}

\begin{itemize}
    \item \textbf{Изолированная система}: $S \to \max$ $\Rightarrow$ $\delta S = 0$, $\delta^2 S < 0$
    \item \textbf{$V, T = \text{const}$}: $F \to \min$ $\Rightarrow$ $\delta F = 0$, $\delta^2 F > 0$
    \item \textbf{$p, T = \text{const}$}: $G \to \min$ $\Rightarrow$ $\delta G = 0$, $\delta^2 G > 0$
    \item \textbf{Адиабата, $p = \text{const}$}: $H \to \min$
    \item \textbf{Адиабата, $V = \text{const}$}: $U \to \min$
\end{itemize}

%═══════════════════════════════════════════════════════════════════
\section{Основные формулы}
%═══════════════════════════════════════════════════════════════════

\subsection{Дифференциалы термодинамических потенциалов}

\textbf{Для обратимых процессов} (знак ``$=$''), для необратимых (знак ``$\leq$''):

\begin{align}
    dU &\leq TdS - pdV \tag{3.1} \\[4pt]
    dF &\leq -pdV - SdT \tag{3.2} \\[4pt]
    dH &\leq TdS + Vdp \tag{3.3} \\[4pt]
    dG &\leq -SdT + Vdp \tag{3.4}
\end{align}

%───────────────────────────────────────────────────────────────────
\subsection{Частные производные потенциалов}

\textbf{Из внутренней энергии $U(V, S)$:}
\begin{equation}
    \boxed{T = \left(\frac{\partial U}{\partial S}\right)_V}, \qquad
    \boxed{p = -\left(\frac{\partial U}{\partial V}\right)_S}
    \tag{3.5}
\end{equation}

\textbf{Из свободной энергии $F(V, T)$:}
\begin{equation}
    \boxed{S = -\left(\frac{\partial F}{\partial T}\right)_V}, \qquad
    \boxed{p = -\left(\frac{\partial F}{\partial V}\right)_T}
    \tag{3.6}
\end{equation}

\textbf{Из энтальпии $H(p, S)$:}
\begin{equation}
    \boxed{T = \left(\frac{\partial H}{\partial S}\right)_p}, \qquad
    \boxed{V = \left(\frac{\partial H}{\partial p}\right)_S}
    \tag{3.7}
\end{equation}

\textbf{Из потенциала Гиббса $G(p, T)$:}
\begin{equation}
    \boxed{S = -\left(\frac{\partial G}{\partial T}\right)_p}, \qquad
    \boxed{V = \left(\frac{\partial G}{\partial p}\right)_T}
    \tag{3.8}
\end{equation}

%───────────────────────────────────────────────────────────────────
\subsection{Соотношения Максвелла}

\begin{equation}
    \boxed{
    \begin{aligned}
        \left(\frac{\partial T}{\partial V}\right)_S &= -\left(\frac{\partial p}{\partial S}\right)_V \\[6pt]
        \left(\frac{\partial S}{\partial V}\right)_T &= \left(\frac{\partial p}{\partial T}\right)_V \\[6pt]
        \left(\frac{\partial T}{\partial p}\right)_S &= \left(\frac{\partial V}{\partial S}\right)_p \\[6pt]
        \left(\frac{\partial S}{\partial p}\right)_T &= -\left(\frac{\partial V}{\partial T}\right)_p
    \end{aligned}
    }
    \tag{3.9}
\end{equation}

%───────────────────────────────────────────────────────────────────
\subsection{Уравнения Гиббса--Гельмгольца}

\begin{equation}
    \boxed{
    \begin{aligned}
        U &= H - p\left(\frac{\partial H}{\partial p}\right)_S = F - T\left(\frac{\partial F}{\partial T}\right)_V \\[6pt]
        G &= F - V\left(\frac{\partial F}{\partial V}\right)_T = H - S\left(\frac{\partial H}{\partial S}\right)_p
    \end{aligned}
    }
    \tag{3.10}
\end{equation}

%───────────────────────────────────────────────────────────────────
\subsection{Неравенство для энтропии}

\begin{equation}
    TdS \geq dU + pdV
    \tag{3.11}
\end{equation}

%───────────────────────────────────────────────────────────────────
\subsection{Адиабатический процесс в переменных $p$, $T$}

\begin{equation}
    \frac{\tilde{C}_p}{T} \cdot \frac{dT}{dp} = \left(\frac{\partial V}{\partial T}\right)_p
    \tag{3.12}
\end{equation}

%───────────────────────────────────────────────────────────────────
\subsection{Эффект Джоуля--Томсона}

Дифференциальный эффект (при $H = \text{const}$):
\begin{equation}
    \tilde{C}_p \, dT + \left[V - T\left(\frac{\partial V}{\partial T}\right)_p\right] dp = 0
    \tag{3.14}
\end{equation}

Коэффициент Джоуля--Томсона:
\begin{equation}
    \left(\frac{\partial T}{\partial p}\right)_H = \frac{T(\partial V/\partial T)_p - V}{\tilde{C}_p}
\end{equation}

Конечная температура:
\begin{equation}
    T_1 = T_0 + \int_{p_1}^{p_2} \frac{1}{\tilde{C}_p}\left[T\left(\frac{\partial V}{\partial T}\right)_p - V\right] dp
    \tag{3.15}
\end{equation}

%═══════════════════════════════════════════════════════════════════
\section{Полезные соотношения из примеров}
%═══════════════════════════════════════════════════════════════════

\subsection{Потенциалы идеального газа ($\nu$ молей)}

\textbf{Свободная энергия:}
\begin{equation}
    F(V,T) = \nu \left[C_V T(1 - \ln T) - RT\ln V - TS_0\right]
\end{equation}

\textbf{Потенциал Гиббса:}
\begin{equation}
    G(p,T) = \nu \left[C_p T(1 - \ln T) - T(S_0 + R\ln V) + RT\right]
\end{equation}

%───────────────────────────────────────────────────────────────────
\subsection{Разность теплоёмкостей}

Для вещества с известным уравнением состояния $V = V(p, T)$:
\begin{equation}
    \boxed{\tilde{C}_p - \tilde{C}_V = -T\left(\frac{\partial V}{\partial p}\right)_T^{-1} \left(\frac{\partial V}{\partial T}\right)_p^2}
\end{equation}

Тождество для частных производных:
\begin{equation}
    \left(\frac{\partial p}{\partial T}\right)_V \left(\frac{\partial T}{\partial V}\right)_p \left(\frac{\partial V}{\partial p}\right)_T = -1
\end{equation}

%───────────────────────────────────────────────────────────────────
\subsection{Полезное соотношение}

\begin{equation}
    \left(\frac{\partial T}{\partial V}\right)_S = -\frac{T(\partial p/\partial T)_V}{\tilde{C}_V}
\end{equation}

%───────────────────────────────────────────────────────────────────
\subsection{Условия механической и термической устойчивости}

\textbf{Механическая устойчивость:}
\begin{equation}
    \left(\frac{\partial p}{\partial V}\right)_T < 0
\end{equation}
(увеличение объёма $\Rightarrow$ уменьшение давления)

\textbf{Термическая устойчивость:}
\begin{equation}
    \tilde{C}_V > 0
\end{equation}
(получение теплоты $\Rightarrow$ увеличение температуры)

%───────────────────────────────────────────────────────────────────
\subsection{Максимальная работа при изотермическом смешении газов}

\begin{equation}
    A_{\max} = -\Delta F = \nu RT_0 \ln\frac{(V_1 + V_2)^2}{V_1 V_2}
\end{equation}

\end{document}
