\documentclass[12pt,a4paper]{article}
\usepackage[utf8]{inputenc}
\usepackage[T2A]{fontenc}
\usepackage[russian]{babel}
\usepackage{amsmath,amssymb,amsfonts}
\usepackage{geometry}
\usepackage{hyperref}
\usepackage{titlesec}
\geometry{margin=2cm}

\titleformat{\part}[display]{\Huge\bfseries\centering}{Часть \thepart}{20pt}{\Huge}
\titlespacing*{\part}{0pt}{50pt}{40pt}

\title{\Huge\textbf{Полная теория и формулы}\\[10pt]\Large Параграфы 4, 7, 12, 13 + Все формулы курса}
\author{}
\date{}

\begin{document}
\maketitle
\tableofcontents
\newpage

%═══════════════════════════════════════════════════════════════════════════════
\part{Полная теория избранных параграфов}
%═══════════════════════════════════════════════════════════════════════════════

%═══════════════════════════════════════════════════════════════════════════════
\section{\S 4. Реальные газы}
%═══════════════════════════════════════════════════════════════════════════════

\subsection{Краткие теоретические сведения}

Зависимость потенциальной энергии взаимодействия $\Phi$ между любыми двумя частицами одноатомного газа от расстояния $r$ между ними показывает, что с уменьшением $r$ на малых расстояниях $\Phi$ растёт, что соответствует силам отталкивания между атомами. Становясь положительной, $\Phi$ очень быстро делается чрезвычайно большой, что соответствует <<непроницаемости>> атомов. На больших расстояниях потенциальная энергия взаимодействия медленно увеличивается, асимптотически приближаясь к нулю. Это соответствует взаимному притяжению атомов. При этом
\[
\Phi_0 = |\min\{\Phi(r)\}| \sim kT_K,
\]
где $k = 1{,}38\cdot 10^{-23}$ Дж/К --- постоянная Больцмана, $T_K$ --- критическая температура данного вещества.

Для реальных газов эмпирически установлено более ста термических уравнений состояния. Наиболее простым из них является \textbf{уравнение Ван-дер-Ваальса}:
\begin{equation}
\boxed{\left(p + \frac{\nu^2 a}{V^2}\right)(V - \nu b) = \nu RT}
\tag{4.1}
\end{equation}

Наличие сил притяжения приводит к появлению дополнительного давления на газ. Действие отталкивания сводится к тому, что молекула не допускает проникновения других молекул в занимаемый ею эффективный объём.

\subsection{Примеры решения задач}

\subsubsection{Пример 4.1. Изотермы газа Ван-дер-Ваальса}

\textbf{Условие:} Изобразить на диаграмме $p-V$ изотермы газа Ван-дер-Ваальса при разных температурах и провести их качественный анализ.

\textbf{Решение:} Среди изотерм одного моля газа Ван-дер-Ваальса при разных значениях $T$ выделяется изотерма, построенная при критической температуре $T_K$, имеющая характерный перегиб в точке $K$. Если $T > T_K$, то вещество существует только в газообразном состоянии. При $T < T_K$ изотермы имеют минимум и максимум, причём на возрастающем участке система неустойчива: малейшая флуктуация объёма, например, в сторону его увеличения, приводит к увеличению давления и к дальнейшему расширению газа. Построенная на основе данных эксперимента изотерма реального газа показана сплошной кривой $ABCD$. Участку $AB$ соответствует жидкость, участку $CD$ --- газ. На прямой $BC$ жидкость и насыщенный пар находятся в динамическом равновесии.

\subsubsection{Пример 4.2. Критические параметры}

\textbf{Условие:} Сосуд объёмом $V = 1{,}5\cdot 10^{-5}$ м$^3$ необходимо наполнить водой при температуре $t = 18$~°C с таким расчётом, чтобы при её нагревании до критической температуры $T_K = 647{,}3$~K в сосуде установилось критическое давление $p_K = 201{,}4\cdot 10^5$~Па. Найти необходимый объём воды.

\textbf{Решение:} В критической точке равны нулю первая и вторая производные от давления по объёму. После дифференцирования уравнения Ван-дер-Ваальса по $V$ получим:
\[
\left(\frac{\partial p}{\partial V}\right)_{T_K} = -\frac{\nu RT_K}{(V_K - \nu b)^2} + \frac{2\nu^2 a}{V_K^3} = 0
\]
\[
\left(\frac{\partial^2 p}{\partial V^2}\right)_{T_K} = \frac{2\nu RT_K}{(V_K - \nu b)^3} - \frac{6\nu^2 a}{V_K^4} = 0
\]

Отсюда можно выразить $a$, $b$ и $\nu$ через $p_K$, $V_K$ и $T_K$:
\begin{equation}
\boxed{a = \frac{27 R^2 T_K^2}{64 p_K}, \qquad b = \frac{RT_K}{8p_K}, \qquad \nu = \frac{8V_K p_K}{3RT_K}}
\end{equation}

Необходимый объём воды: $V_1 = \mu\nu/\rho = 8\mu V_K p_K/(3RT_K\rho) = 3\cdot 10^{-6}$~м$^3$.

\subsubsection{Пример 4.3. Внутренняя энергия газа Ван-дер-Ваальса}

\textbf{Условие:} Доказать, что теплоёмкость $\tilde{C}_V$ газа Ван-дер-Ваальса не зависит от объёма $V$, а может зависеть только от температуры $T$. Найти выражение для внутренней энергии $U$.

\textbf{Решение:} Теплоёмкость $\tilde{C}_V = T(\partial S/\partial T)_V$ не будет зависеть от объёма $V$, если производная $(\partial \tilde{C}_V/\partial V)_T = T[\partial^2 S/(\partial T\partial V)]$ окажется равной нулю. Воспользовавшись одним из соотношений Максвелла (3.9), последнее соотношение можно переписать в виде:
\[
\left(\frac{\partial \tilde{C}_V}{\partial V}\right)_T = T\left(\frac{\partial^2 p}{\partial T^2}\right)_V
\]

В равенстве нулю $(\partial^2 p/\partial T^2)_V$ легко непосредственно убедиться, дважды продифференцировав уравнение Ван-дер-Ваальса.

Перепишем соотношение (2.7) в виде $(\partial U/\partial V)_T = T(\partial p/\partial T)_V - p$ и подставим в него $p$ из уравнения Ван-дер-Ваальса. В итоге получим:
\[
\left(\frac{\partial U}{\partial V}\right)_T = \frac{\nu^2 a}{V^2}
\]

Если $\tilde{C}_V = (\partial U/\partial T)_V$ зависит только от температуры, то $U = -\nu^2 a/V + \int \tilde{C}_V(T)dT$. В частном случае $\tilde{C}_V = \nu C_V$, где $C_V = \text{const}$:
\begin{equation}
\boxed{U = -\frac{\nu^2 a}{V} + \nu C_V T}
\end{equation}

\subsubsection{Пример 4.4. Адиабатическое расширение в пустоту}

\textbf{Условие:} Один моль двухатомного газа Ван-дер-Ваальса, константа $a$ которого известна, адиабатически расширяется в пустоту от объёма $V_1$ до объёма $V_2$. Найти изменение температуры газа.

\textbf{Решение:} При адиабатическом расширении в пустоту газ не получает тепло и не совершает работу. Следовательно, его внутренняя энергия остаётся постоянной. Используя результаты предыдущего примера:
\[
C_V(T_2 - T_1) - \frac{a}{V_2} + \frac{a}{V_1} = 0
\]

Для двухатомного газа $C_V = 5R/2$, откуда:
\begin{equation}
\boxed{T_1 - T_2 = \frac{2a(V_2 - V_1)}{5RV_1V_2}}
\end{equation}

\subsubsection{Пример 4.5. Смешивание газов Ван-дер-Ваальса}

\textbf{Условие:} Два сосуда с теплоизолированными стенками, имеющие объёмы $V_1$ и $V_2$, соединены трубкой с краном. При закрытом кране в каждом из них при температуре $T_1$ находится по одному молю одного и того же газа Ван-дер-Ваальса. Кран открывают. Определить температуру $T_2$ и давление $p$ после установления равновесия.

\textbf{Решение:} При решении этой задачи нельзя пользоваться законом Дальтона, справедливым только для идеального газа. По условию стенки сосудов теплоизолированы, и при смешивании газы не совершают работу против внешних сил. Поэтому изменение внутренней энергии всей системы равно нулю:
\[
\Delta U = 2C_V T_2 - \frac{4a}{V_1 + V_2} - 2C_V T_1 + \frac{a}{V_1} + \frac{a}{V_2} = 0
\]

Откуда:
\begin{equation}
\boxed{T_2 = T_1 - \frac{a(V_2 - V_1)^2}{2C_V V_1 V_2(V_2 + V_1)}}
\end{equation}

Подставляя $T_2$ в уравнение Ван-дер-Ваальса:
\begin{equation}
\boxed{p = \frac{2RT_2}{V_1 + V_2 - 2b} - \frac{4a}{(V_1 + V_2)^2}}
\end{equation}

\subsubsection{Пример 4.6. Разность теплоёмкостей}

\textbf{Условие:} Найти $C_p - C_V$ для одного моля газа Ван-дер-Ваальса.

\textbf{Решение:} Воспользуемся формулой $C_p - C_V = T(\partial p/\partial T)_V(\partial V/\partial T)_p$. Входящие частные производные найдём из уравнения Ван-дер-Ваальса:
\[
\left(\frac{\partial T}{\partial V}\right)_p = \frac{TRV^3 - 2a(V - b)^2}{RV^3(V - b)}, \qquad \left(\frac{\partial p}{\partial T}\right)_V = \frac{R}{V - b}
\]

Подставляя их в выражение для разности теплоёмкостей:
\begin{equation}
\boxed{C_p - C_V = R\left[1 - \frac{2a(V-b)^2}{RTV^3}\right]^{-1}}
\end{equation}

Разность $C_p - C_V$ для газа Ван-дер-Ваальса оказывается больше, чем соответствующая величина у идеального газа.

\subsubsection{Пример 4.7. Энтропия газа Ван-дер-Ваальса}

\textbf{Условие:} Найти выражение для энтропии $\nu$ молей газа Ван-дер-Ваальса.

\textbf{Решение:} Согласно определению, энтропия выражается формулой $dS = dU/T + (p/T)dV$. Подставим дифференциал внутренней энергии $dU = (\nu^2 a/V^2)dV + \nu C_V dT$ из примера 4.3 и давление $p$ газа Ван-дер-Ваальса. После приведения подобных членов:
\[
dS = \frac{\nu R\,dV}{V - b} + \frac{\nu C_V\,dT}{T}
\]

Интегрируя:
\begin{equation}
\boxed{S = \nu C_V \ln T + \nu R \ln(V - \nu b) + S_0}
\end{equation}

Энтропия газа Ван-дер-Ваальса меньше энтропии идеального газа для одних и тех же $\nu$, $V$ и $T$. Это объясняется тем, что из-за собственного объёма молекул неопределённость в их расположении меньше, чем у идеального газа.

\subsubsection{Пример 4.9. Эффект Джоуля--Томсона}

\textbf{Условие:} Оценить изменение температуры в процессе Джоуля--Томсона для газа Ван-дер-Ваальса.

\textbf{Решение:} Найдём вид соотношения $(\partial T/\partial p)_H = [T(\partial V/\partial T)_p - V]/C_p$ для газа Ван-дер-Ваальса. Используя уравнение Ван-дер-Ваальса:
\[
\left(\frac{\partial T}{\partial V}\right)_p = \frac{T}{V - b} - \frac{2a}{RV^2} + \frac{2ab}{RV^3}
\]

Подставляя и упрощая при $b \ll V$ и $a/V^2 \ll p \approx RT/V$:
\[
\left(\frac{\partial T}{\partial p}\right)_H = \frac{2a/RT - b}{C_p}
\]

Из этого выражения видно, что изменение температуры газа Ван-дер-Ваальса при необратимом адиабатическом расширении обусловлено отличием его свойств от свойств идеального газа.

\textbf{Температура инверсии:}
\begin{equation}
\boxed{T^* = \frac{2a}{bR}}
\end{equation}

Если $T < T^*$, газ охлаждается при расширении; если $T > T^*$ --- нагревается. Для большинства газов $T^*$ значительно выше комнатной температуры, поэтому они в процессе Джоуля--Томсона охлаждаются (азот, кислород). Температура инверсии водорода и гелия ниже комнатной, и при $T_0$ они нагреваются.

%═══════════════════════════════════════════════════════════════════════════════
\section{\S 7. Распределения Гиббса}
%═══════════════════════════════════════════════════════════════════════════════

\subsection{Краткие теоретические сведения}

Состояние термодинамической системы из $N$ частиц определяется совокупностью координат и импульсов: $X = (x_1, x_2, \ldots, x_N) = (\vec{r}_1, \vec{r}_2, \ldots, \vec{r}_N, \vec{p}_1, \vec{p}_2, \ldots, \vec{p}_N)$.

\textbf{Уравнения Гамильтона:}
\begin{equation}
\boxed{\frac{d\vec{r}_i}{dt} = \frac{\partial H(X)}{\partial \vec{p}_i}, \qquad \frac{d\vec{p}_i}{dt} = -\frac{\partial H(X)}{\partial \vec{r}_i}}
\tag{7.1}
\end{equation}

\textbf{Функция Гамильтона:}
\begin{equation}
\boxed{H(X) = \sum_{i=1}^{N} \frac{\vec{p}_i^2}{2m} + U(\vec{r}_1, \ldots, \vec{r}_N) = \sum_{i=1}^{N} \left(\frac{\vec{p}_i^2}{2m} + U_0(\vec{r}_i)\right) + \sum_{i<j} \Phi(|\vec{r}_i - \vec{r}_j|)}
\tag{7.2}
\end{equation}

\textbf{Уравнение Лиувилля:}
\begin{equation}
\frac{\partial w_N}{\partial t} + \sum_{i=1}^{N} \left(\frac{p_i}{m} \cdot \frac{\partial w_N}{\partial \vec{r}_i} - \frac{\partial U_0}{\partial \vec{r}_i} \cdot \frac{\partial w_N}{\partial \vec{p}_i} - \frac{\partial}{\partial \vec{r}_i} \sum_{j=1}^{N} \Phi(|\vec{r}_i - \vec{r}_j|) \cdot \frac{\partial w_N}{\partial \vec{p}_i}\right) = 0
\tag{7.3}
\end{equation}

\subsubsection{Микроканоническое распределение Гиббса}

Для изолированной системы с энергией $E$:
\begin{equation}
\boxed{w_N(X, a, E) = \frac{\delta(H(X, a) - E)}{\tilde{\Omega}(a, E)}}
\tag{7.4}
\end{equation}

где $\tilde{\Omega}(a, E) = \int \delta(H(X, a) - E)\,dX$ --- статистический вес.

\subsubsection{Каноническое распределение Гиббса}

Для системы в термостате:
\begin{equation}
\boxed{w_N(X, a, T) = \frac{\exp(-H(X, a)/kT)}{Z(a, T)}}
\tag{7.5}
\end{equation}

где $Z(a, T) = \int \exp(-H(X, a)/kT)\,dX$ --- статистический интеграл.

\subsubsection{Квантовое каноническое распределение}

\begin{equation}
\boxed{P_n(N, a, T) = \frac{\exp(-E_n/kT)}{Z_1} = \exp\left(\frac{F - E_n}{kT}\right)}
\tag{7.8}
\end{equation}

\textbf{Статистическая сумма:}
\begin{equation}
\boxed{Z_1 = \sum_n \exp(-E_n/kT)}
\tag{7.9}
\end{equation}

\textbf{Связь со свободной энергией:}
\begin{equation}
\boxed{F = -kT \ln Z_1}
\tag{7.10}
\end{equation}

\subsubsection{Большое каноническое распределение}

Для системы с переменным числом частиц:
\begin{equation}
\boxed{P_{n,N}(a, T, \mu) = \frac{\exp[(\mu N - E_{n,N})/kT]}{Z_2}}
\tag{7.12}
\end{equation}

\textbf{Большая статистическая сумма:}
\begin{equation}
\boxed{Z_2 = \sum_{N,n} \exp[(\mu N - E_{n,N})/kT]}
\tag{7.13}
\end{equation}

\textbf{Большой термодинамический потенциал:}
\begin{equation}
\boxed{\Omega = -kT \ln Z_2 = F - \mu N}
\tag{7.14}
\end{equation}

\subsubsection{Теорема о равномерном распределении энергии}

\begin{equation}
\boxed{\bar{H} = \left(\frac{s_1}{2} + \frac{s_2}{\eta}\right)kT}
\tag{7.21}
\end{equation}

где $s_1$ --- полное число степеней свободы, $s_2$ --- число колебательных степеней свободы.

\subsection{Примеры решения задач}

\subsubsection{Пример 7.1. Плотность распределения для одной молекулы}

\textbf{Условие:} Для идеального одноатомного газа записать плотность распределения вероятностей $w_1(\vec{r}, \vec{p})$ для координат и импульсов одной молекулы. Газ состоит из $N$ молекул, находится в сосуде объёмом $V$ и имеет температуру $T$. Масса молекулы $m$.

\textbf{Решение:} Для идеального одноатомного газа в сосуде объёмом $V$ функция Гамильтона принимает вид:
\[
H(X) = \sum_{i=1}^{N} \left(\frac{p_i^2}{2m} + U_0(\vec{r}_i)\right),
\]
где $U_0(\vec{r}_i) = 0$, если $\vec{r}_i \in V$, и $U_0(\vec{r}_i) = \infty$, если $\vec{r}_i \notin V$.

Используя формулу гауссова интеграла
\[
\int_{-\infty}^{\infty} u^{2n} \exp(-\alpha u^2)\,du = \frac{(2n-1)!!}{(2\alpha)^n} \sqrt{\frac{\pi}{\alpha}},
\]
вычислим статистический интеграл:
\[
Z(a, T) = \int\exp(-H(X, a)/kT)\,dX = V^N (2\pi mkT)^{3N/2}
\]

Искомая функция:
\begin{equation}
\boxed{w_1(x_i) = \frac{\exp(-\vec{p}_i^2/2mkT)}{V(2\pi mkT)^{3/2}}}
\tag{7.23}
\end{equation}

Распределение частиц в пространстве $\int w_1(\vec{r}, \vec{p})\,d\vec{p} = 1/V$ равномерно внутри объёма $V$, а распределение частиц по импульсам является трёхмерным распределением Гаусса.

\subsubsection{Пример 7.2. Теорема о равномерном распределении}

\textbf{Условие:} Пользуясь каноническим распределением Гиббса, доказать формулу (7.21) при $s_1 = 3N$ и $\eta = 2$, т.е. $\bar{H} = 3NkT$.

\textbf{Решение:} Пусть гамильтониан термодинамической системы из $N$ частиц:
\[
H = \sum_{i=1}^{N} \sum_{\alpha=1}^{3} \left(\frac{p_{i\alpha}^2}{2m} + \frac{A_{i\alpha} q_{i\alpha}^2}{2}\right),
\]
где $A_{i\alpha}$ --- положительные константы. Тогда $\overline{p_{i\alpha} \partial H/\partial p_{i\alpha}} = \overline{q_{i\alpha} \partial H/\partial q_{i\alpha}} = kT$.

По определению среднего значения и используя гауссов интеграл, получаем:
\[
\overline{p_{i\alpha} \frac{\partial H}{\partial p_{i\alpha}}} = (2\pi mkT)^{-1/2} \int_{-\infty}^{\infty} \frac{p_{i\alpha}^2}{m} \exp\left(-\frac{p_{i\alpha}^2}{2mkT}\right) dp_{i\alpha} = kT
\]

Среднее значение гамильтониана:
\[
\bar{H} = \sum_{i=1}^{N} \sum_{\alpha=1}^{3} \left(\frac{\overline{p_{i\alpha}^2}}{2m} + \frac{A_{i\alpha}\overline{q_{i\alpha}^2}}{2}\right) = \frac{1}{2} \sum_{i=1}^{N} \sum_{\alpha=1}^{3} \left[\overline{p_{i\alpha} \frac{\partial H}{\partial p_{i\alpha}}} + \overline{q_{i\alpha} \frac{\partial H}{\partial q_{i\alpha}}}\right] = 3NkT
\]

\subsubsection{Пример 7.3. Внутренняя энергия двухатомного газа}

\textbf{Условие:} Найти внутреннюю энергию одного моля идеального двухатомного газа, модель которого представляет молекулы из двух материальных точек, связанных пружиной с жёсткостью $\kappa$.

\textbf{Решение:} Каждая молекула имеет шесть степеней свободы ($s_1 = 6N_A$). Из них одна --- колебательная ($s_2 = N_A$). Потенциальная энергия $\kappa(\lambda\xi_i)^2/2 = \lambda^2\kappa\xi_i^2/2$ удовлетворяет условию $\eta = 2$.

По формуле (7.21):
\begin{equation}
\boxed{\bar{H} = \sum_{i=1}^{N_A} \bar{H}_i = N_A\left(\frac{s_1}{2} + \frac{s_2}{\eta}\right)kT = \frac{7N_A kT}{2} = \frac{7RT}{2}}
\end{equation}

\subsubsection{Пример 7.4. Термодинамика идеального газа}

\textbf{Условие:} Найти уравнение состояния, внутреннюю энергию, энтропию и химический потенциал $\nu$ молей одноатомного идеального газа.

\textbf{Решение:} Для идеального одноатомного газа классический статистический интеграл $Z(a, T) = V^N(2\pi mkT)^{3N/2}$ (см. пример 7.1).

В соответствии с (7.10), (7.11): $F = -kT \ln Z_1$, где $Z_1 = [(2\pi\hbar)^{3N} N!]^{-1}Z$. Используя формулу Стирлинга $N! \sim N^N \exp(-N)$:
\begin{equation}
\boxed{F = -kTN\left\{1 + \ln\left[\frac{V(2\pi mkT)^{3/2}}{N(2\pi\hbar)^3}\right]\right\}}
\tag{7.32}
\end{equation}

\textbf{Внутренняя энергия:} $U = F - T(\partial F/\partial T)_V = 3kTN/2$.

\textbf{Давление:} $p = -(\partial F/\partial V)_T = kTN/V$, откуда $pV = \nu RT$.

\textbf{Энтропия (формула Сакура--Тетроде):}
\begin{equation}
\boxed{S = -\left(\frac{\partial F}{\partial T}\right)_{V,N} = kN\left\{\frac{5}{2} + \ln\left[\frac{(2\pi mkT)^{3/2}V}{(2\pi\hbar)^3 N}\right]\right\}}
\tag{7.33}
\end{equation}

\textbf{Химический потенциал:} Используя формулу $\mu = (\partial U/\partial N)_{T,V} - T(\partial S/\partial N)_{T,V}$:
\begin{equation}
\boxed{\mu = kT \ln\left[\frac{N(2\pi\hbar)^3}{V(2\pi mkT)^{3/2}}\right]}
\end{equation}

\subsubsection{Пример 7.5. Газ в поле тяжести}

\textbf{Условие:} Вычислить теплоёмкость при постоянном объёме идеального одноатомного газа из $N$ частиц, находящегося при температуре $T$ в поле силы тяжести. Газ находится в полубесконечном прямом цилиндре с площадью основания $s$.

\textbf{Решение:} В цилиндрической системе координат функция Гамильтона:
\[
H = \sum_{i=1}^{N} \left(\frac{p_i^2}{2m} + U_0(\rho_i) + mgz_i\right),
\]
где $U_0(\rho_i) = 0$, если $\rho_i \in s$, и $U_0(\rho_i) = \infty$, если $\rho_i \notin s$.

Статистический интеграл:
\[
Z_1(T, V) = \frac{(2\pi mkT)^{3N/2}}{(2\pi\hbar)^{3N}N!} \left[\iint_s dx_i dy_i \int_0^\infty \exp(-mgz_i/kT)\,dz_i\right]^N = \frac{(2\pi mkT)^{3N/2} s^N (kT/mg)^N}{(2\pi\hbar)^{3N}N!}
\]

Свободная энергия:
\[
F = -kTN\left\{1 + \ln\left[\frac{(2\pi mkT)^{3/2}skT}{(2\pi\hbar)^3 mgN}\right]\right\}
\]

Теплоёмкость $\tilde{C}_V = -T(\partial^2 F/\partial T^2)_V$:
\begin{equation}
\boxed{\tilde{C}_V = \frac{5kN}{2}}
\end{equation}

Учёт потенциальной энергии гравитационного поля приводит к увеличению теплоёмкости газа (по сравнению с $3kN/2$ для свободного газа).

\subsubsection{Пример 7.6. Газ в центрифуге}

\textbf{Условие:} Найти изменение свободной энергии идеального газа, находящегося в центрифуге при температуре $T$, вызванное её вращением с угловой скоростью $\omega$. Радиус центрифуги $R$, высота $h_0$.

\textbf{Решение:} Во вращающейся системе отсчёта функция Гамильтона:
\[
H = \sum_{i=1}^{N} \left(\frac{p_i^2}{2m} + U_0(z_i) - \frac{m\omega^2\rho_i^2}{2}\right)
\]

Статистический интеграл и свободная энергия:
\[
F = -kTN\left\{1 + \ln\left[\frac{(2\pi mkT)^{3/2}}{(2\pi\hbar)^3N}\right] + \ln\left[\frac{2\pi h_0 kT}{m\omega^2} \left(\exp\left(\frac{m\omega^2 R^2}{2kT}\right) - 1\right)\right]\right\}
\]

При больших скоростях вращения ($\omega^2 \gg 2kT/mR^2$):
\begin{equation}
\boxed{\Delta F \approx kTN\left[\ln\frac{m\omega^2 R^2}{2kT} - \frac{m\omega^2 R^2}{2kT}\right]}
\end{equation}

\subsubsection{Пример 7.7. Ультрарелятивистский газ}

\textbf{Условие:} Найти теплоёмкость и уравнение состояния идеального газа в ультрарелятивистском случае (энергия частицы $E = cp$, где $c$ --- скорость света).

\textbf{Решение:} Классический статистический интеграл:
\[
Z = \int\exp(-H/kT)\,dX = \left(4\pi \int_0^\infty p^2 \exp(-cp/kT)\,dp \cdot \int d\vec{r}\right)^N = [8\pi V(kT/c)^3]^N
\]

Свободная энергия с учётом формулы Стирлинга:
\[
F = -kTN\left\{1 + \ln\left[\frac{8\pi V(kT)^3}{N(2\pi\hbar)^3 c^3}\right]\right\}
\]

Теплоёмкость:
\begin{equation}
\boxed{\tilde{C}_V = -T\left(\frac{\partial^2 F}{\partial T^2}\right)_V = 3Nk = 3\nu R}
\end{equation}

Теплоёмкость в два раза превышает теплоёмкость нерелятивистского одноатомного газа.

Давление: $p = -(\partial F/\partial V)_{T,N} = kTN/V$, или $pV = \nu RT$ --- такое же, как у нерелятивистского газа.

%═══════════════════════════════════════════════════════════════════════════════
\section{\S 12. Флуктуации в равновесных системах}
%═══════════════════════════════════════════════════════════════════════════════

\subsection{Краткие теоретические сведения}

\textbf{Флуктуации} --- самопроизвольные отклонения функций динамических переменных $B_j(X)$ от их средних значений $\bar{B}_j$.

\textbf{Дисперсия:}
\[
(B_j - \bar{B}_j)^2 = (\Delta B_j)^2 = \overline{B_j^2} - (\bar{B}_j)^2 = \sigma^2_{B_j}
\]

\textbf{Относительная флуктуация:} $\delta_{B_j} = \sigma_{B_j}/\bar{B}_j$

\textbf{Коэффициент корреляции:} $K_{ij} = \overline{\Delta B_i \cdot \Delta B_j}$

\subsubsection{Формула Эйнштейна}

Вероятность перехода в квазиравновесное состояние:
\begin{equation}
\boxed{w = C \exp\left(-\frac{\Delta U + p\Delta V - \mu\Delta N - T\Delta S}{kT}\right)}
\tag{12.2}
\end{equation}

\textbf{Частные случаи:}

Изолированная система ($U, V, N = \text{const}$):
\begin{equation}
w = C \exp(\Delta S/k)
\tag{12.3}
\end{equation}

Система в термостате ($T, V, N = \text{const}$):
\begin{equation}
w = C \exp(-\Delta F/kT)
\tag{12.4}
\end{equation}

Система с переменным числом частиц ($T, V, \mu = \text{const}$):
\begin{equation}
w = C \exp(-\Delta\Omega/kT)
\tag{12.5}
\end{equation}

\textbf{Общая формула для вероятности флуктуаций:}
\begin{equation}
\boxed{w = C \exp\left(\frac{\Delta p \Delta V - \Delta\mu \Delta N - \Delta T \Delta S}{2kT}\right)}
\tag{12.6}
\end{equation}

\subsection{Примеры решения задач}

\subsubsection{Пример 12.1. Флуктуации энергии}

\textbf{Условие:} Для системы с переменным числом частиц доказать равенство $kT^2(\partial U/\partial T)_{V,N} = \overline{H^2} - \bar{U}^2$, где $U$ --- среднее значение гамильтониана $H(X)$.

\textbf{Решение:} Продифференцируем по $T$ при постоянных $V$ и $N$ внутреннюю энергию системы:
\[
U = \int H(X)\exp\{[\Omega + \mu N - H(X)]/kT\}\,dX
\]

Умножим $(\partial U/\partial T)_{V,N}$ на $kT^2$:
\[
kT^2\left(\frac{\partial U}{\partial T}\right)_{V,N} = kT^2 \int H(X)\exp\left[\frac{\Omega + \mu N - H(X)}{kT}\right] \times \left[\frac{1}{kT}\left(\frac{\partial \Omega}{\partial T}\right)_{V,\mu} - \frac{\Omega + \mu N - H(X)}{kT^2}\right]dX
\]
\[
= U\left[T\left(\frac{\partial \Omega}{\partial T}\right)_{V,\mu} - (\Omega + \mu N)\right] + \overline{H^2(X)}
\]

После подстановки $(\partial \Omega/\partial T)_V = -S$ и учёта равенства $\Omega = U - TS - \mu N$:
\begin{equation}
\boxed{kT^2\left(\frac{\partial U}{\partial T}\right)_V = \overline{H^2}(X) - U^2}
\end{equation}

\subsubsection{Пример 12.2. Дисперсия числа частиц}

\textbf{Условие:} Используя большое каноническое распределение Гиббса, доказать равенство $(\Delta N)^2 = \overline{N^2} - (\bar{N})^2 = kT(\partial \bar{N}/\partial \mu)_{T,V}$.

\textbf{Решение:} Среднее число частиц $\bar{N}$ и среднее значение квадрата числа частиц $\overline{N^2}$:
\[
\bar{N} = \tilde{Z}_2^{-1} \sum_N \frac{N}{(2\pi\hbar)^{3N} N!} \int\exp\left[\frac{\mu N - H(X, N, a)}{kT}\right]dX = \frac{kT}{\tilde{Z}_2} \frac{\partial \tilde{Z}_2}{\partial \mu}
\]
\[
\overline{N^2} = \frac{(kT)^2}{\tilde{Z}_2} \frac{\partial^2 \tilde{Z}_2}{\partial \mu^2}
\]

Продифференцируем соотношение (7.16) по $\mu$ при постоянных $T$ и $V$, умножим на $kT$ и учтём, что $\Omega = -kT \ln \tilde{Z}_2$:
\[
kT\left(\frac{\partial \bar{N}}{\partial \mu}\right)_{T,V} = -kT \frac{\partial^2 \Omega}{\partial \mu^2} = \frac{(kT)^2}{\tilde{Z}_2}\left[\frac{\partial^2 \tilde{Z}_2}{\partial \mu^2} - \frac{1}{\tilde{Z}_2}\left(\frac{\partial \tilde{Z}_2}{\partial \mu}\right)^2\right] = \overline{N^2} - (\bar{N})^2
\]

\begin{equation}
\boxed{(\Delta N)^2 = kT\left(\frac{\partial \bar{N}}{\partial \mu}\right)_{T,V}}
\tag{12.10}
\end{equation}

\subsubsection{Пример 12.3. Коэффициент корреляции $\Delta N \Delta H$}

\textbf{Условие:} Выразить коэффициент корреляции между флуктуациями энергии и числа частиц через уравнение состояния и дисперсию числа частиц.

\textbf{Решение:} Продифференцируем $\bar{N}$ по $T$ при постоянных $V$ и $\mu$:
\[
\left(\frac{\partial \bar{N}}{\partial T}\right)_{V,\mu} = \frac{-\bar{U}\bar{N} + \mu(\bar{N})^2 - \mu\overline{N^2} + \overline{HN}}{kT^2}
\]

Коэффициент корреляции:
\[
\overline{\Delta N \Delta H} = \overline{(N - \bar{N})(H - U)} = \overline{NH} - \bar{U}\bar{N} = kT^2\left(\frac{\partial \bar{N}}{\partial T}\right)_{V,\mu} + \mu(\overline{N^2} - (\bar{N})^2)
\]
\[
= kT^2\left(\frac{\partial \bar{N}}{\partial T}\right)_{V,\mu} + kT\mu\left(\frac{\partial \bar{N}}{\partial \mu}\right)_{T,V}
\]

Используя соотношение $0 = (\partial \bar{N}/\partial T)_{V,\mu}(\partial T/\partial \mu)_{V,\bar{N}} + (\partial \bar{N}/\partial \mu)_{T,V}$:
\begin{equation}
\boxed{\overline{\Delta N \Delta H} = (\Delta N)^2 \left(\frac{\partial U}{\partial N}\right)_{V,T}}
\end{equation}

Для идеального газа $(\Delta N)^2 = \bar{N}$, $U = 3kT\bar{N}/2$, и, следовательно, $\overline{\Delta N \Delta H} = 3kT\bar{N}/2$.

\subsubsection{Пример 12.4. Дисперсия объёма}

\textbf{Условие:} В цилиндре под поршнем под давлением $p$ находится $N$ молекул идеального газа при температуре $T$. Рассчитать дисперсию объёма газа.

\textbf{Решение:} Изменение давления газа равно $\Delta p = (\partial p/\partial V)_{T,N}\Delta V$, т.к. $\Delta T = 0$, $\Delta N = 0$. При этом формула (12.6) принимает вид:
\[
w = C \exp\left[\frac{(\partial p/\partial V)_{T,N}(\Delta V)^2}{2kT}\right]
\]

В соответствии с формулой (12.1):
\begin{equation}
\boxed{(\Delta V)^2 = -kT\left(\frac{\partial V}{\partial p}\right)_{T,N} = N\left(\frac{kT}{p}\right)^2 \cdot \frac{1}{N} = \frac{V^2}{N}}
\end{equation}

\subsubsection{Пример 12.5. Дисперсия числа частиц (квазитермодинамика)}

\textbf{Условие:} В цилиндре объёмом $V$ при температуре $T$ находится переменное число частиц $N$, задаваемое химическим потенциалом $\mu$. Рассчитать дисперсию числа частиц газа.

\textbf{Решение:} По условию $\Delta V = \Delta T = 0$, а $\Delta\mu = (\partial\mu/\partial N)_{V,T}\Delta N$. При этом формула (12.6) принимает вид:
\[
w(\Delta N) = C \exp\left[-\frac{(\partial\mu/\partial N)_{V,T}(\Delta N)^2}{2kT}\right]
\]

В соответствии с формулой (12.1):
\begin{equation}
\boxed{(\Delta N)^2 = kT\left(\frac{\partial N}{\partial \mu}\right)_{V,T}}
\end{equation}

\subsubsection{Пример 12.6. Дисперсия внутренней энергии идеального газа}

\textbf{Условие:} Найти дисперсию внутренней энергии $U(V, T)$ идеального одноатомного газа, состоящего из $N$ частиц.

\textbf{Решение:} Усреднив очевидное равенство:
\[
(\Delta U)^2 = \left(\frac{\partial U}{\partial T}\right)_V^2(\Delta T)^2 + 2\left(\frac{\partial U}{\partial T}\right)_V\left(\frac{\partial U}{\partial V}\right)_T\overline{\Delta T\Delta V} + \left(\frac{\partial U}{\partial V}\right)_T^2(\Delta V)^2
\]

В переменных $T$ и $V$:
\[
\Delta S = \left(\frac{\partial S}{\partial T}\right)_V\Delta T + \left(\frac{\partial S}{\partial V}\right)_T\Delta V, \qquad \Delta p = \left(\frac{\partial p}{\partial T}\right)_V\Delta T + \left(\frac{\partial p}{\partial V}\right)_T\Delta V
\]

С учётом соотношений Максвелла, формула (12.6) принимает вид:
\[
w = C \exp\left[-\frac{(\partial S/\partial T)_V(\Delta T)^2 + (\partial p/\partial V)_T(\Delta V)^2}{2kT}\right]
\]

Откуда следует:
\[
(\Delta T)^2 = kT\left(\frac{\partial T}{\partial S}\right)_V, \qquad (\Delta V)^2 = -kT\left(\frac{\partial V}{\partial p}\right)_T, \qquad \overline{\Delta T\Delta V} = 0
\]

Подставляя $(\Delta T)^2$ и $(\Delta V)^2$ в выражение для $(\Delta U)^2$:
\[
(\Delta U)^2 = kT\left[\left(\frac{\partial U}{\partial T}\right)_V^2\left(\frac{\partial T}{\partial S}\right)_{V,N} - \left(\frac{\partial p}{\partial V}\right)_{T,N}\left(\frac{\partial U}{\partial V}\right)_T^2\right]
\]

Для одноатомного идеального газа $U = 3NkT/2$, $(\partial U/\partial V)_T = 0$:
\begin{equation}
\boxed{(\Delta U)^2 = \frac{3N(kT)^2}{2}}
\end{equation}

%═══════════════════════════════════════════════════════════════════════════════
\section{\S 13. Кинетическое уравнение Больцмана и процессы переноса}
%═══════════════════════════════════════════════════════════════════════════════

\subsection{Краткие теоретические сведения}

\subsubsection{Функции распределения}

Связь $S$-частичной функции распределения с $N$-частичной:
\begin{equation}
w_S(x_1, x_2, \ldots, x_S, t) = V^S \int w_N(x_1, x_2, \ldots, x_N, t)\,dx_{S+1}dx_{S+2}\ldots dx_N
\tag{13.1}
\end{equation}

\textbf{Внутренняя энергия:}
\begin{equation}
U(t) = \frac{N}{V}\int\frac{\vec{p}^2}{2m}w_1(\vec{r}, \vec{p}, t)\,d\vec{p}\,d\vec{r} + \frac{N^2}{2V^2}\int\Phi(|\vec{r}_1 - \vec{r}_2|)w_2\,d\vec{p}_1d\vec{r}_1d\vec{p}_2d\vec{r}_2
\tag{13.3}
\end{equation}

\subsubsection{Цепочка уравнений ББГКИ}

Первое уравнение цепочки:
\begin{equation}
\left(\frac{\partial}{\partial t} + \frac{\vec{p}_1}{m}\frac{\partial}{\partial \vec{r}_1} - \frac{\partial U_0(\vec{r}_1)}{\partial \vec{r}_1}\frac{\partial}{\partial \vec{p}_1}\right)w_1 = \frac{N}{V}\int\frac{\partial\Phi(|\vec{r}_1 - \vec{r}_2|)}{\partial \vec{r}_1}\frac{\partial w_2}{\partial \vec{p}_1}\,d\vec{r}_2d\vec{p}_2
\tag{13.4}
\end{equation}

\subsubsection{Корреляционные функции}

\begin{equation}
g_2(x_1, x_2, t) = w_2(x_1, x_2, t) - w_1(x_1, t)w_1(x_2, t)
\tag{13.6}
\end{equation}

\subsubsection{Кинетическое уравнение Больцмана}

\begin{multline}
\left(\frac{\partial}{\partial t} + \frac{\vec{p}_1}{m}\frac{\partial}{\partial \vec{r}_1} - \frac{\partial U_0(\vec{r}_1)}{\partial \vec{p}_1}\frac{\partial}{\partial \vec{p}_1}\right)w_1(\vec{r}_1, \vec{p}_1, t) = \\
= \frac{N}{mV}\int_0^\infty dp_2\rho d\rho \int_0^{2\pi} d\varphi|\vec{p}_2 - \vec{p}_1|[w_1(\vec{r}_1, \vec{p}_2', t)w_1(\vec{r}_1, \vec{p}_1', t) - w_1(\vec{r}_1, \vec{p}_2, t)w_1(\vec{r}_1, \vec{p}_1, t)]
\tag{13.8}
\end{multline}

\subsubsection{Равновесное распределение Максвелла--Больцмана}

\begin{equation}
\boxed{w_1^{(0)} = \frac{1}{(2\pi mkT)^{3/2}} \exp\left(-\frac{p^2 + 2mU_0(\vec{r})}{kT}\right)\left(\int\exp(-U_0(\vec{r})/kT)\,d\vec{r}\right)^{-1}}
\tag{13.9}
\end{equation}

\subsubsection{Релаксационное приближение}

\begin{equation}
\boxed{\left(\frac{\partial}{\partial t} + \frac{\vec{p}}{m}\frac{\partial}{\partial \vec{r}} - \frac{\partial U_0(\vec{r})}{\partial \vec{r}}\frac{\partial}{\partial \vec{p}}\right)w_1 = -\frac{w_1 - w_1^{(0)}}{\tau}}
\tag{13.10}
\end{equation}

\subsection{Процессы переноса}

\subsubsection{Диффузия}

\textbf{Закон Фика:}
\begin{equation}
\boxed{u(x, t)n(x, t) = -D\frac{\partial n}{\partial x}}
\tag{13.11}
\end{equation}

\textbf{Уравнение диффузии:}
\begin{equation}
\boxed{\frac{\partial n}{\partial t} = D\frac{\partial^2 n}{\partial x^2}}
\tag{13.12}
\end{equation}

\textbf{Решение для бесконечной среды:}
\begin{equation}
n(x, t) = (4\pi Dt)^{-1/2} \int_{-\infty}^{\infty} \tilde{n}(x')\exp\left[-\frac{(x - x')^2}{4Dt}\right]dx'
\tag{13.13}
\end{equation}

\subsubsection{Вязкость}

\begin{equation}
\boxed{f_y = -\eta\frac{\partial v_y}{\partial x}}
\tag{13.14}
\end{equation}

где $\eta$ --- коэффициент вязкости.

\subsubsection{Теплопроводность}

\textbf{Закон Фурье:}
\begin{equation}
\boxed{J_x = -\kappa\frac{\partial T}{\partial x}}
\tag{13.15}
\end{equation}

\textbf{Уравнение теплопроводности:}
\begin{equation}
\boxed{c_v\rho\frac{\partial T}{\partial t} = \kappa\Delta T + q}
\tag{13.16}
\end{equation}

\subsection{Примеры решения задач}

\subsubsection{Пример 13.1. Выражение термодинамических функций через $w_1$}

\textbf{Условие:} Выразить через одночастичную функцию распределения концентрацию частиц, плотность газа, среднюю скорость их упорядоченного движения и локальную температуру. Масса одной частицы $m$.

\textbf{Решение:}

\textbf{Концентрация частиц:}
\[
n(\vec{r}, t) = \overline{\sum_{i=1}^{N} \delta(\vec{r} - \vec{r}_i(t))} = \frac{N}{V}\int w_1(\vec{r}, \vec{p}, t)\,d\vec{p}
\]

\textbf{Плотность массы:} $\rho(\vec{r}, t) = mn(\vec{r}, t)$

\textbf{Скорость упорядоченного движения:}
\[
\vec{u}(\vec{r}, t)n(\vec{r}, t) = \frac{N}{mV}\int \vec{p} w_1(\vec{r}, \vec{p}, t)\,d\vec{p}
\]

\textbf{Локальная температура:}
\[
\frac{3n(\vec{r}, t)kT(\vec{r}, t)}{2} = \frac{mN}{2V}\int \left[\frac{\vec{p}}{m} - \vec{u}(\vec{r}, t)\right]^2 w_1(\vec{r}, \vec{p}, t)\,d\vec{p}
\]

\subsubsection{Пример 13.2. Вывод первого уравнения цепочки ББГКИ}

\textbf{Условие:} Используя уравнение Лиувилля (7.3), получить первое уравнение цепочки уравнений ББГКИ.

\textbf{Решение:} Умножим уравнение Лиувилля на $V$ и проинтегрируем по $x_2, x_3, \ldots, x_N$.

Член с производной по времени: $\partial w_1(\vec{r}_1, \vec{p}_1, t)/\partial t$.

При интегрировании по пространственным координатам выделим первое слагаемое:
\[
\frac{V}{m}\int\sum_{i=1}^{N} \vec{p}_i\left[\frac{\partial w_N}{\partial \vec{r}_i}\right]dx_2 dx_3\ldots dx_N = \frac{\vec{p}_1}{m}\frac{\partial w_1}{\partial \vec{r}_1}
\]
(остальные $N-1$ слагаемых равны нулю из-за граничных условий).

Аналогично для членов с $\partial U_0/\partial \vec{r}_i$ остаётся только:
\[
-\frac{\partial U_0(\vec{r}_1)}{\partial \vec{r}_1}\frac{\partial w_1}{\partial \vec{p}_1}
\]

Члены взаимодействия дают правую часть:
\[
\frac{N}{V}\int\frac{\partial\Phi(|\vec{r}_1 - \vec{r}_2|)}{\partial \vec{r}_1}\frac{\partial w_2(x_1, x_2, t)}{\partial \vec{p}_1}\,dx_2
\]

Собирая всё вместе, получаем первое уравнение цепочки ББГКИ (13.4).

\subsubsection{Пример 13.3. Длина свободного пробега}

\textbf{Условие:} Считая молекулы газа абсолютно упругими шариками массой $m$ и диаметром $\sigma$, оценить среднее время между двумя последовательными столкновениями $\tau_0$ и среднюю длину свободного пробега $l_0$. Концентрация частиц $n$, температура газа $T$.

\textbf{Решение:} Допустим, что все молекулы, кроме одной, движущейся со скоростью $v_0$, покоятся. Движущаяся молекула, пройдя расстояние $d$, столкнётся со всеми неподвижными молекулами, центры которых находятся в круглом прямом цилиндре с радиусом основания $\sigma$ и высотой $d$.

Средняя длина свободного пробега равна высоте такого цилиндра, в котором в среднем находится одна молекула: $\pi\sigma^2 l_0 n = 1$, откуда:
\begin{equation}
\boxed{l_0 = \frac{1}{\pi\sigma^2 n}}
\end{equation}

Среднее время между столкновениями (положив $v_0$ равным среднему значению модуля скорости относительного движения двух молекул):
\begin{equation}
\boxed{\tau_0 = \frac{l_0}{v_0} = \sqrt{\frac{m}{8\pi kT\sigma^4 n^2}}}
\end{equation}

\subsubsection{Пример 13.4. H-теорема Больцмана}

\textbf{Условие:} Доказать, что в ходе эволюции замкнутой системы введённая Больцманом энтропия $S_1 = -k\int w_1(\vec{r}, \vec{p}, t)\ln w_1(\vec{r}, \vec{p}, t)\,d\vec{r}\,d\vec{p}$ возрастает или остаётся неизменной.

\textbf{Решение:} Продифференцируем $S_1$ по времени:
\[
\frac{dS_1}{dt} = -k\int\left[\frac{\partial w_1}{\partial t}\ln w_1\right]dx - k\int w_1\frac{1}{w_1}\frac{\partial w_1}{\partial t}\,dx
\]

Первое слагаемое с $(\partial/\partial t)\int w_1 dx = 0$ из-за нормировки. Интегралы с $(\vec{p}/m)(\partial w_1/\partial \vec{r})$ и $(\partial U_0/\partial \vec{r})(\partial w_1/\partial \vec{p})$ равны нулю при интегрировании по частям с использованием граничных условий.

Подставим интеграл столкновений в форме Боголюбова и проведём симметризацию:
\[
\frac{dS_1}{dt} = \frac{kN}{4mV}\int d\vec{p}_2 d\vec{p} d\vec{r}\int_0^\infty \rho d\rho\int_0^{2\pi} d\varphi|\vec{p}_2 - \vec{p}| \times
\]
\[
\times \ln\frac{w_1(\vec{r}, \vec{p}, t)w_1(\vec{r}, \vec{p}_2, t)}{w_1(\vec{r}, \vec{p}', t)w_1(\vec{r}, \vec{p}_2', t)} \cdot \{w_1(\vec{r}, \vec{p}', t)w_1(\vec{r}, \vec{p}_2', t) - w_1(\vec{r}, \vec{p}, t)w_1(\vec{r}, \vec{p}_2, t)\}
\]

Произведение логарифма на фигурную скобку имеет структуру $(B_1 - B_2)\ln(B_1/B_2)$ и при любых положительных $B_1$ и $B_2$ положительно:
\begin{equation}
\boxed{\frac{dS_1}{dt} \geq 0}
\end{equation}

\subsubsection{Пример 13.5. Коэффициент диффузии}

\textbf{Условие:} С помощью кинетического уравнения Больцмана с релаксационным членом (13.10) оценить величину коэффициента диффузии газа. Газ находится вблизи равновесия, изменение концентрации стационарно, внешние поля отсутствуют.

\textbf{Решение:} В силу стационарности и отсутствия внешних полей, первое и третье слагаемые в (13.10) обращаются в ноль:
\[
w_1(\vec{r}, \vec{p}) = w_1^{(0)} - \tau\frac{\vec{p}}{m}\frac{\partial w_1(\vec{r}, \vec{p})}{\partial \vec{r}}
\]

Вблизи равновесия заменим $w_1$ на $w_1^{(0)}$ в правой части. Умножим на $Np_x/mV$ и проинтегрируем по $\vec{p}$:
\[
u(x, t)n(x, t) = \frac{N}{V}\int\frac{p_x}{m}w_1^{(0)}d\vec{p} - \frac{\tau N}{V}\frac{\partial}{\partial x}\int\frac{p_x^2}{m^2}w_1^{(0)}d\vec{p}
\]

Первый интеграл равен нулю (в равновесии поток отсутствует). Второй даёт:
\[
u(x, t)n(x, t) = -\frac{\tau kT}{m}\frac{\partial n(x, t)}{\partial x}
\]

Сравнивая с законом Фика (13.11):
\begin{equation}
\boxed{D = \frac{\tau kT}{m}}
\end{equation}

\subsubsection{Пример 13.6. Вывод уравнения теплопроводности}

\textbf{Условие:} Исходя из выражения (13.15) для потока тепла, получить уравнение теплопроводности (13.16).

\textbf{Решение:} Выделим в среде небольшой параллелепипед с гранями $x = x_0$, $x = x_0 + \Delta x$, $y = y_0$, $y = y_0 + \Delta y$, $z = z_0$, $z = z_0 + \Delta z$.

Через грань $x = x_0$ за время $dt$ поступает теплота $dQ_1 = -\Delta y\Delta z\,dt\,\kappa\,\partial T/\partial x|_{x=x_0}$, через грань $x = x_0 + \Delta x$ уходит $dQ_2 = -\Delta y\Delta z\,dt\,\kappa\,\partial T/\partial x|_{x=x_0+\Delta x}$.

Уравнение теплового баланса:
\[
c_v\rho\Delta x\Delta y\Delta z\,dT = dQ_1 - dQ_2 + dQ_3 - dQ_4 + dQ_5 - dQ_6 + q\Delta x\Delta y\Delta z\,dt
\]
\[
= \kappa\Delta x\Delta y\Delta z\,dt\left(\frac{\partial^2 T}{\partial x^2} + \frac{\partial^2 T}{\partial y^2} + \frac{\partial^2 T}{\partial z^2}\right) + q\Delta x\Delta y\Delta z\,dt
\]

Деля на $\Delta x\Delta y\Delta z\,dt$:
\begin{equation}
\boxed{c_v\rho\frac{\partial T}{\partial t} = \kappa\left(\frac{\partial^2 T}{\partial x^2} + \frac{\partial^2 T}{\partial y^2} + \frac{\partial^2 T}{\partial z^2}\right) + q}
\end{equation}

\subsubsection{Пример 13.7. Теплообмен между сосудами}

\textbf{Условие:} Два сосуда с газом нагреты до температур $T_{10}$ и $T_{20}$ ($T_{10} > T_{20}$). Массы газов $m_1$, $m_2$, удельные теплоёмкости $c_1$, $c_2$. Сосуды соединены металлическим стержнем длиной $L$, площадью $S$, коэффициентом теплопроводности $\kappa$. Найти зависимость разности температур от времени.

\textbf{Решение:} Так как теплоёмкость стержня равна нулю, уравнение (13.16) принимает вид: $\partial^2 T(x, t)/\partial x^2 = 0$.

Его решение с граничными условиями $T(0, t) = T_1(t)$, $T(L, t) = T_2(t)$:
\[
T(x, t) = T_1(t) + x[T_2(t) - T_1(t)]/L
\]

Поток тепла постоянен вдоль стержня:
\[
J_x = -\kappa\frac{\partial T}{\partial x} = \frac{\kappa[T_1(t) - T_2(t)]}{L} = -\frac{c_1 m_1}{S}\frac{dT_1}{dt} = \frac{c_2 m_2}{S}\frac{dT_2}{dt}
\]

Уравнение для разности температур:
\[
\frac{d[T_2(t) - T_1(t)]}{dt} = -\frac{\kappa S}{L}\left(\frac{1}{c_1 m_1} + \frac{1}{c_2 m_2}\right)[T_2(t) - T_1(t)]
\]

Решение:
\begin{equation}
\boxed{T_2(t) - T_1(t) = (T_{20} - T_{10})\exp\left[-\frac{\kappa St}{L}\left(\frac{1}{c_1 m_1} + \frac{1}{c_2 m_2}\right)\right]}
\end{equation}

%═══════════════════════════════════════════════════════════════════════════════
\part{Полный конспект формул}
%═══════════════════════════════════════════════════════════════════════════════

%═══════════════════════════════════════════════════════════════════════════════
\section{\S 1. Основные понятия и первое начало термодинамики}
%═══════════════════════════════════════════════════════════════════════════════

\subsection{Фундаментальные константы}
\begin{align}
N_A &= 6{,}02 \cdot 10^{23} \text{ моль}^{-1} && \text{(число Авогадро)} \\
R &= 8{,}31 \text{ Дж/(моль$\cdot$К)} && \text{(универсальная газовая постоянная)} \\
k &= \frac{R}{N_A} = 1{,}38 \cdot 10^{-23} \text{ Дж/К} && \text{(постоянная Больцмана)}
\end{align}

\subsection{Уравнение Клапейрона--Менделеева}
\begin{equation}
\boxed{pV = \frac{m}{\mu}RT = \nu RT}
\tag{1.2}
\end{equation}

\subsection{Первое начало термодинамики}
\begin{equation}
\boxed{\delta Q = dU + \delta A}
\tag{1.8}
\end{equation}

\subsection{Работа газа}
\begin{equation}
\boxed{A = \int_{V_0}^{V_1} p(V) \, dV}
\tag{1.6}
\end{equation}

\subsection{Теплоёмкости идеального газа}
\begin{equation}
\boxed{C_V = \frac{i}{2}R, \qquad C_p = C_V + R = \frac{i+2}{2}R, \qquad \gamma = \frac{C_p}{C_V} = \frac{i+2}{i}}
\end{equation}

\subsection{Политропный процесс}
\begin{equation}
\boxed{pV^n = \text{const}}
\tag{1.11}
\end{equation}

\subsection{КПД цикла Карно}
\begin{equation}
\boxed{\eta = 1 - \frac{T_2}{T_1}}
\end{equation}

%═══════════════════════════════════════════════════════════════════════════════
\section{\S 2. Второе начало термодинамики. Энтропия}
%═══════════════════════════════════════════════════════════════════════════════

\subsection{Дифференциал энтропии}
\begin{equation}
\boxed{dS \equiv \frac{\delta Q}{T} = \frac{dU}{T} + \frac{p\,dV}{T}}
\tag{2.1}
\end{equation}

\subsection{Изменение энтропии идеального газа}
\begin{equation}
\boxed{S_2 - S_1 = R \ln\frac{V_2}{V_1} + C_V \ln\frac{T_2}{T_1}}
\tag{2.4}
\end{equation}

\subsection{Неравенство Клаузиуса}
\begin{equation}
\boxed{\oint \frac{\delta Q}{T} \leq 0}
\tag{2.5}
\end{equation}

%═══════════════════════════════════════════════════════════════════════════════
\section{\S 3. Термодинамические потенциалы}
%═══════════════════════════════════════════════════════════════════════════════

\subsection{Определения}
\begin{center}
\begin{tabular}{|c|c|c|}
\hline
Потенциал & Определение & Естественные переменные \\
\hline
Внутренняя энергия $U$ & --- & $V$, $S$ \\
Свободная энергия $F$ & $U - TS$ & $V$, $T$ \\
Энтальпия $H$ & $U + pV$ & $p$, $S$ \\
Потенциал Гиббса $G$ & $U - TS + pV$ & $p$, $T$ \\
\hline
\end{tabular}
\end{center}

\subsection{Дифференциалы потенциалов}
\begin{align}
dU &\leq TdS - pdV \\
dF &\leq -pdV - SdT \\
dH &\leq TdS + Vdp \\
dG &\leq -SdT + Vdp
\end{align}

\subsection{Соотношения Максвелла}
\begin{equation}
\boxed{
\begin{aligned}
\left(\frac{\partial T}{\partial V}\right)_S &= -\left(\frac{\partial p}{\partial S}\right)_V \\[6pt]
\left(\frac{\partial S}{\partial V}\right)_T &= \left(\frac{\partial p}{\partial T}\right)_V \\[6pt]
\left(\frac{\partial T}{\partial p}\right)_S &= \left(\frac{\partial V}{\partial S}\right)_p \\[6pt]
\left(\frac{\partial S}{\partial p}\right)_T &= -\left(\frac{\partial V}{\partial T}\right)_p
\end{aligned}
}
\tag{3.9}
\end{equation}

%═══════════════════════════════════════════════════════════════════════════════
\section{\S 5. Системы с переменным числом частиц}
%═══════════════════════════════════════════════════════════════════════════════

\subsection{Химический потенциал}
\begin{equation}
\boxed{\mu = \left(\frac{\partial U}{\partial N}\right)_{S,V} = \left(\frac{\partial F}{\partial N}\right)_{T,V} = \left(\frac{\partial G}{\partial N}\right)_{T,p}}
\tag{5.6}
\end{equation}

\subsection{Уравнение Клапейрона--Клаузиуса}
\begin{equation}
\boxed{\frac{dp}{dT} = \frac{\lambda}{T(\nu_1 - \nu_2)}}
\tag{5.18}
\end{equation}

%═══════════════════════════════════════════════════════════════════════════════
\section{\S 6. Некоторые вероятностные представления}
%═══════════════════════════════════════════════════════════════════════════════

\subsection{Среднее значение и дисперсия}
\begin{equation}
\boxed{\bar{x} = \int_a^b x\, w^{(1)}(x)\,dx, \qquad \sigma^2(x) = \int_a^b (x - \bar{x})^2 w^{(1)}(x)\,dx}
\end{equation}

\subsection{Распределение Гаусса}
\begin{equation}
\boxed{w^{(1)}(x) = \frac{1}{\sqrt{2\pi}\sigma} \exp\left(-\frac{(x - x_0)^2}{2\sigma^2}\right)}
\tag{6.11}
\end{equation}

\subsection{Биномиальное распределение}
\begin{equation}
\boxed{P_N(n) = C_N^n \left(\frac{v}{V}\right)^n \left(1 - \frac{v}{V}\right)^{N-n}}
\tag{6.14}
\end{equation}

%═══════════════════════════════════════════════════════════════════════════════
\section{\S 8. Распределение Максвелла}
%═══════════════════════════════════════════════════════════════════════════════

\subsection{Распределение по импульсам}
\begin{equation}
\boxed{w^{(3)}(\vec{p}) = (2\pi mkT)^{-3/2} \exp\left(-\frac{\vec{p}^2}{2mkT}\right)}
\tag{8.1}
\end{equation}

\subsection{Характерные скорости}
\begin{align}
\text{Наивероятнейшая:} \quad &v^* = \sqrt{\frac{2kT}{m}} \\[6pt]
\text{Средняя:} \quad &\bar{v} = \sqrt{\frac{8kT}{\pi m}} \\[6pt]
\text{Среднеквадратичная:} \quad &\sqrt{\overline{v^2}} = \sqrt{\frac{3kT}{m}}
\end{align}

\subsection{Средняя энергия молекулы}
\begin{equation}
\boxed{\bar{E} = \frac{3kT}{2}}
\tag{8.6}
\end{equation}

\subsection{Число ударов о стенку}
\begin{equation}
\boxed{N = \frac{Sn\tau\bar{v}}{4}}
\tag{8.10}
\end{equation}

%═══════════════════════════════════════════════════════════════════════════════
\section{\S 9. Распределение Больцмана}
%═══════════════════════════════════════════════════════════════════════════════

\subsection{Распределение по координатам}
\begin{equation}
\boxed{w^{(3)}(\vec{r}) = \frac{\exp(-U_0(\vec{r})/kT)}{\displaystyle\int_V \exp(-U_0(\vec{r})/kT)\,d\vec{r}}}
\tag{9.1}
\end{equation}

\subsection{Барометрическая формула}
\begin{equation}
\boxed{n(z) = n(0) \exp\left(-\frac{m_0 gz}{kT}\right), \qquad p(z) = p(0) \exp\left(-\frac{m_0 gz}{kT}\right)}
\end{equation}

\subsection{Центрифугирование}
\begin{equation}
\boxed{\frac{n_i(r)}{n_i(0)} = \exp\left(\frac{m_i \omega^2 r^2}{2kT}\right)}
\tag{9.10}
\end{equation}

%═══════════════════════════════════════════════════════════════════════════════
\section{\S 10. Цепочка уравнений для равновесных функций распределения}
%═══════════════════════════════════════════════════════════════════════════════

\subsection{Уравнение состояния с поправкой}
\begin{equation}
\boxed{p = \frac{NkT}{V}\left[1 + \frac{N(\tilde{b} - \tilde{a}/kT)}{V}\right]}
\tag{10.22}
\end{equation}

\subsection{Связь с константами Ван-дер-Ваальса}
\begin{equation}
\boxed{a = N_A^2 \tilde{a}, \qquad b = N_A \tilde{b}}
\tag{10.23}
\end{equation}

%═══════════════════════════════════════════════════════════════════════════════
\section{\S 11. Идеальные квантовые газы}
%═══════════════════════════════════════════════════════════════════════════════

\subsection{Распределение Бозе--Эйнштейна}
\begin{equation}
\boxed{\bar{N}_l = \frac{1}{\exp[(E_l - \mu)/kT] - 1}}
\tag{11.8}
\end{equation}

\subsection{Распределение Ферми--Дирака}
\begin{equation}
\boxed{\bar{N}_l = \frac{1}{\exp[(E_l - \mu)/kT] + 1}}
\tag{11.9}
\end{equation}

\subsection{Энергия Ферми}
\begin{equation}
\boxed{E_F = \frac{(3\pi^2 \bar{N}/V)^{2/3} \hbar^2}{2m}}
\end{equation}

%═══════════════════════════════════════════════════════════════════════════════
\section{\S 14. Броуновское движение}
%═══════════════════════════════════════════════════════════════════════════════

\subsection{Уравнение Фоккера--Планка}
\begin{equation}
\boxed{\left(\frac{\partial}{\partial t} + \frac{\vec{p}}{M}\frac{\partial}{\partial \vec{r}} - \frac{\partial U_0(\vec{r})}{\partial \vec{r}}\frac{\partial}{\partial \vec{p}}\right)w_1 = \gamma MkT\frac{\partial^2 w_1}{\partial \vec{p}^2} + \frac{\partial}{\partial \vec{p}}[\gamma\vec{p}w_1]}
\tag{14.1}
\end{equation}

\subsection{Коэффициент диффузии}
\begin{equation}
\boxed{D = \frac{kT}{M\gamma} = \frac{kT}{6\pi\eta_0 R_0}}
\end{equation}

\subsection{Средний квадрат смещения}
\begin{equation}
\boxed{\overline{r^2} = 6Dt} \quad \text{(при } t \gg \gamma^{-1}\text{)}
\tag{14.6}
\end{equation}

\subsection{Формула Эйнштейна для дисперсии импульса}
\begin{equation}
\boxed{(\Delta p)^2 = 2MkT\gamma t} \quad \text{(при } \tau_1 \ll t \ll \gamma^{-1}\text{)}
\end{equation}

\end{document}
