\documentclass[12pt,a4paper]{article}
\usepackage[utf8]{inputenc}
\usepackage[T2A]{fontenc}
\usepackage[russian]{babel}
\usepackage{amsmath,amssymb,amsfonts}
\usepackage{geometry}
\geometry{margin=2cm}

\title{\S 6. Некоторые вероятностные представления}
\author{Конспект формул}
\date{}

\begin{document}
\maketitle

%═══════════════════════════════════════════════════════════════════
\section{Краткая теория}
%═══════════════════════════════════════════════════════════════════

\subsection{Вероятность и случайные величины}

\textbf{Вероятность события} $A$: $P(A) = m/n$, где $m$ --- число реализаций события $A$, $n$ --- полное число возможных равновероятных реализаций.

\textbf{Дискретная случайная величина} принимает счётное число значений $x_1, x_2, \ldots, x_n$ с вероятностями $P(x_i)$, причём $\sum_i P(x_i) = 1$.

\textbf{Непрерывная случайная величина} $x$ характеризуется \textbf{плотностью вероятности} $w^{(1)}(x)$, где $w^{(1)}(x)dx$ --- вероятность попадания $x$ в интервал $(x, x+dx)$.

Условие нормировки: $\int_a^b w^{(1)}(x)dx = 1$.

\subsection{Распределение Гаусса}

Нормальный закон распределения (распределение Гаусса) --- важнейшее распределение в статистической физике. Характеризуется двумя параметрами: средним значением $x_0$ и дисперсией $\sigma^2$.

\subsection{Статистическая независимость}

Случайные величины $x_1, x_2, \ldots, x_n$ \textbf{независимы}, если их совместная плотность вероятности равна произведению плотностей вероятности каждой из них.

%═══════════════════════════════════════════════════════════════════
\section{Основные формулы}
%═══════════════════════════════════════════════════════════════════

\subsection{Функция распределения}

\begin{equation}
    \boxed{F_1(\tilde{x}) = \int_a^{\tilde{x}} w^{(1)}(x)\,dx}
    \tag{6.1}
\end{equation}

Принимает значение, равное вероятности того, что $x$ попадает в интервал $(a, \tilde{x})$.

%───────────────────────────────────────────────────────────────────
\subsection{Среднее значение}

\textbf{Дискретная величина:}
\begin{equation}
    \boxed{\bar{x} = \sum_{i=1}^{n} x_i P(x_i)}
    \tag{6.2a}
\end{equation}

\textbf{Непрерывная величина:}
\begin{equation}
    \boxed{\bar{x} = \int_a^b x\, w^{(1)}(x)\,dx}
    \tag{6.2b}
\end{equation}

%───────────────────────────────────────────────────────────────────
\subsection{Среднее значение функции}

\textbf{Дискретный случай:}
\begin{equation}
    \boxed{\overline{f(x)} = \sum_{i=1}^{n} f(x_i) P(x_i)}
    \tag{6.3a}
\end{equation}

\textbf{Непрерывный случай:}
\begin{equation}
    \boxed{\overline{f(x)} = \int_a^b f(x)\, w^{(1)}(x)\,dx}
    \tag{6.3b}
\end{equation}

%───────────────────────────────────────────────────────────────────
\subsection{Дисперсия}

\textbf{Дискретная величина:}
\begin{equation}
    \boxed{\sigma^2(x) = \sum_{i=1}^{n} (x_i - \bar{x})^2 P(x_i)}
    \tag{6.4a}
\end{equation}

\textbf{Непрерывная величина:}
\begin{equation}
    \boxed{\sigma^2(x) = \int_a^b (x - \bar{x})^2 w^{(1)}(x)\,dx}
    \tag{6.4b}
\end{equation}

\textbf{Стандартное отклонение:} $\sigma(x) = \sqrt{\sigma^2(x)}$

\textbf{Относительная флуктуация:}
\begin{equation}
    \boxed{\delta(x) = \frac{\sigma(x)}{\bar{x}}}
\end{equation}

%───────────────────────────────────────────────────────────────────
\subsection{Среднее функции многих переменных}

\begin{equation}
    \boxed{\overline{f(x_1, \ldots, x_n)} = \int\ldots\int w^{(n)}(x_1, \ldots, x_n) f(x_1, \ldots, x_n)\,dx_1\ldots dx_n}
    \tag{6.5}
\end{equation}

%───────────────────────────────────────────────────────────────────
\subsection{Маргинальное распределение}

\begin{equation}
    \boxed{w^{(1)}(x_m) = \int\ldots\int w^{(n)}(x_1, \ldots, x_n)\,dx_1\ldots dx_{m-1}\,dx_{m+1}\ldots dx_n}
    \tag{6.6}
\end{equation}

%───────────────────────────────────────────────────────────────────
\subsection{Преобразование плотности вероятности}

При замене переменных $y_i = y_i(x_1, \ldots, x_n)$:
\begin{equation}
    \boxed{w^{(n)}(y_1, \ldots, y_n) = w^{(n)}(x_1(y), \ldots, x_n(y)) \left|\frac{\partial(x_1, \ldots, x_n)}{\partial(y_1, \ldots, y_n)}\right|}
    \tag{6.7}
\end{equation}

Последний множитель --- якобиан преобразования.

\textbf{Для одной переменной} $y = y(x)$:
\begin{equation}
    w^{(1)}(y) = w^{(1)}(x(y))\left|\frac{dx}{dy}\right|
\end{equation}

%───────────────────────────────────────────────────────────────────
\subsection{Формула полной вероятности}

\begin{equation}
    \boxed{P(A) = \sum_{i=1}^{n} P(B_i) P(A|B_i)}
    \tag{6.8}
\end{equation}

где $B_1, B_2, \ldots, B_n$ --- попарно несовместные события, образующие полную группу; $P(A|B_i)$ --- условная вероятность события $A$ при реализации $B_i$.

%───────────────────────────────────────────────────────────────────
\subsection{Статистическая независимость}

\begin{equation}
    \boxed{w^{(n)}(x_1, x_2, \ldots, x_n) = w^{(1)}(x_1) w^{(1)}(x_2) \cdots w^{(1)}(x_n)}
    \tag{6.9}
\end{equation}

\textbf{Следствие:} для независимых величин $\overline{x_1 x_2} = \bar{x}_1 \cdot \bar{x}_2$.

%───────────────────────────────────────────────────────────────────
\subsection{Распределение Гаусса (нормальное)}

\begin{equation}
    \boxed{w^{(1)}(x) = \frac{1}{\sqrt{2\pi}\sigma} \exp\left(-\frac{(x - x_0)^2}{2\sigma^2}\right)}
    \tag{6.11}
\end{equation}

\textbf{Свойства:}
\begin{itemize}
    \item Наиболее вероятное значение: $x^* = x_0$
    \item Среднее значение: $\bar{x} = x_0$
    \item Дисперсия: $\sigma^2(x) = \sigma^2$
    \item Относительная флуктуация: $\delta(x) = \sigma/x_0$
\end{itemize}

%═══════════════════════════════════════════════════════════════════
\section{Полезные соотношения из примеров}
%═══════════════════════════════════════════════════════════════════

\subsection{Биномиальное распределение}

Вероятность того, что $n$ из $N$ частиц окажутся в объёме $v$ ($v \ll V$):
\begin{equation}
    \boxed{P_N(n) = C_N^n \left(\frac{v}{V}\right)^n \left(1 - \frac{v}{V}\right)^{N-n} = \frac{N!}{n!(N-n)!} \left(\frac{v}{V}\right)^n \left(1 - \frac{v}{V}\right)^{N-n}}
    \tag{6.14}
\end{equation}

\textbf{Среднее число частиц:}
\begin{equation}
    \boxed{\bar{n} = \frac{Nv}{V}}
\end{equation}

\textbf{Дисперсия:}
\begin{equation}
    \boxed{\sigma^2(n) = \left(1 - \frac{v}{V}\right)\frac{Nv}{V}}
    \tag{6.15}
\end{equation}

\textbf{Относительная флуктуация:}
\begin{equation}
    \boxed{\delta = \frac{\sqrt{V/v - 1}}{\sqrt{N}}}
\end{equation}

%───────────────────────────────────────────────────────────────────
\subsection{Предельные случаи биномиального распределения}

\textbf{Распределение Пуассона} (при $N \gg 1$, $v \ll V$):
\begin{equation}
    \boxed{P_N(n) \approx \frac{(Nv/V)^n}{n!} \exp\left(-\frac{Nv}{V}\right) = \frac{\bar{n}^n}{n!} e^{-\bar{n}}}
\end{equation}

\textbf{Гауссова аппроксимация} (в окрестности максимума):
\begin{equation}
    \boxed{P_N(n) = \frac{1}{\sqrt{2\pi}} \cdot \frac{1}{\sqrt{(1 - v/V)Nv/V}} \cdot \exp\left[-\frac{(n - Nv/V)^2}{2(1 - v/V)Nv/V}\right]}
    \tag{6.16}
\end{equation}

%───────────────────────────────────────────────────────────────────
\subsection{Поведение флуктуаций с ростом $N$}

\begin{itemize}
    \item Среднее число частиц $\bar{n} \propto N$
    \item Дисперсия $\sigma^2 \propto N$
    \item Относительная флуктуация $\delta \propto 1/\sqrt{N}$ (уменьшается!)
\end{itemize}

Это фундаментальный результат статистической физики: относительные флуктуации макроскопических величин исчезающе малы для больших систем.

\end{document}
