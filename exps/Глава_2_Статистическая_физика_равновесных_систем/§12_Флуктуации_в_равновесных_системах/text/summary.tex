\documentclass[12pt,a4paper]{article}
\usepackage[utf8]{inputenc}
\usepackage[T2A]{fontenc}
\usepackage[russian]{babel}
\usepackage{amsmath,amssymb,amsfonts}
\usepackage{geometry}
\geometry{margin=2cm}

\title{\S 12. Флуктуации в равновесных системах}
\author{Конспект формул}
\date{}

\begin{document}
\maketitle

%═══════════════════════════════════════════════════════════════════
\section{Краткая теория}
%═══════════════════════════════════════════════════════════════════

\subsection{Определение флуктуаций}

Флуктуации --- самопроизвольные отклонения динамических переменных $B_j(X)$ от их средних значений $\bar{B}_j$.

\textbf{Дисперсия (квадратичная флуктуация):}
\begin{equation}
    \boxed{(B_j - \bar{B}_j)^2 = (\Delta B_j)^2 = \overline{B_j^2} - (\bar{B}_j)^2 = \sigma_{B_j}^2}
\end{equation}

\textbf{Относительная флуктуация:} $\delta_{B_j} = \sigma_{B_j}/\bar{B}_j$

\textbf{Коэффициент корреляции:} $K_{ij} = \overline{\Delta B_i \Delta B_j}$

\subsection{Квазитермодинамическая теория}

Малая часть системы также может характеризоваться термодинамическими параметрами. Флуктуации рассматриваются как медленные переходы между квазиравновесными состояниями. Флуктуации в различных микрообластях считаются независимыми.

%═══════════════════════════════════════════════════════════════════
\section{Основные формулы}
%═══════════════════════════════════════════════════════════════════

\subsection{Распределение Гаусса для флуктуаций}

\begin{equation}
    \boxed{w(B_j) = \left(\frac{\lambda}{2\pi}\right)^{1/2} \exp\left(-\frac{\lambda B_j^2}{2}\right)}
    \tag{12.1}
\end{equation}

При этом дисперсия: $(\Delta B_j)^2 = 1/\lambda$.

%───────────────────────────────────────────────────────────────────
\subsection{Формула Эйнштейна}

Вероятность перехода в квазиравновесное состояние:
\begin{equation}
    \boxed{w = C \exp\left(-\frac{\Delta U + p\Delta V - \mu\Delta N - T\Delta S}{kT}\right)}
    \tag{12.2}
\end{equation}

где $C$ --- нормировочная константа.

%───────────────────────────────────────────────────────────────────
\subsection{Частные случаи формулы (12.2)}

\textbf{Изолированная система} (постоянные $U$, $V$, $N$):
\begin{equation}
    \boxed{w = C \exp(\Delta S/k)}
    \tag{12.3}
\end{equation}

\textbf{Система в термостате} (постоянные $T$, $V$, $N$):
\begin{equation}
    \boxed{w = C \exp(-\Delta F/kT)}
    \tag{12.4}
\end{equation}

\textbf{Система с переменным числом частиц} (постоянные $T$, $V$, $\mu$):
\begin{equation}
    \boxed{w = C \exp(-\Delta\Omega/kT)}
    \tag{12.5}
\end{equation}

%───────────────────────────────────────────────────────────────────
\subsection{Общая формула для вероятности флуктуаций}

\begin{equation}
    \boxed{w = C \exp\left(\frac{\Delta p \Delta V - \Delta\mu \Delta N - \Delta T \Delta S}{2kT}\right)}
    \tag{12.6}
\end{equation}

%───────────────────────────────────────────────────────────────────
\subsection{Флуктуации энергии}

\begin{equation}
    \boxed{kT^2 \left(\frac{\partial U}{\partial T}\right)_{V,N} = \overline{H^2} - \bar{U}^2}
    \tag{12.7}
\end{equation}

%───────────────────────────────────────────────────────────────────
\subsection{Флуктуации числа частиц}

Среднее число частиц:
\begin{equation}
    \boxed{\bar{N} = \frac{kT}{\tilde{Z}_2} \frac{\partial \tilde{Z}_2}{\partial \mu}}
    \tag{12.8}
\end{equation}

Среднее квадрата числа частиц:
\begin{equation}
    \boxed{\overline{N^2} = \frac{(kT)^2}{\tilde{Z}_2} \frac{\partial^2 \tilde{Z}_2}{\partial \mu^2}}
    \tag{12.9}
\end{equation}

\textbf{Дисперсия числа частиц:}
\begin{equation}
    \boxed{(\Delta N)^2 = \overline{N^2} - (\bar{N})^2 = kT \left(\frac{\partial \bar{N}}{\partial \mu}\right)_{T,V}}
    \tag{12.10}
\end{equation}

%───────────────────────────────────────────────────────────────────
\subsection{Корреляция энергии и числа частиц}

\begin{equation}
    \boxed{\overline{\Delta N \Delta H} = kT^2 \left(\frac{\partial \bar{N}}{\partial T}\right)_{V,\mu} + kT\mu \left(\frac{\partial \bar{N}}{\partial \mu}\right)_{T,V}}
    \tag{12.12}
\end{equation}

Эквивалентная форма:
\begin{equation}
    \boxed{\overline{\Delta N \Delta H} = (\Delta N)^2 \left(\frac{\partial U}{\partial N}\right)_{V,T}}
    \tag{12.13}
\end{equation}

%───────────────────────────────────────────────────────────────────
\subsection{Флуктуации температуры и объёма}

При постоянном числе частиц из формулы (12.15):
\begin{equation}
    \boxed{w = C \exp\left(-\frac{(\partial S/\partial T)_V (\Delta T)^2 - (\partial p/\partial V)_T (\Delta V)^2}{2kT}\right)}
    \tag{12.15}
\end{equation}

Следствия:
\begin{equation}
    \boxed{(\Delta T)^2 = kT \left(\frac{\partial T}{\partial S}\right)_V = \frac{kT^2}{\tilde{C}_V}}
\end{equation}

\begin{equation}
    \boxed{(\Delta V)^2 = -kT \left(\frac{\partial V}{\partial p}\right)_T}
\end{equation}

\begin{equation}
    \boxed{\overline{\Delta T \Delta V} = 0}
\end{equation}

%───────────────────────────────────────────────────────────────────
\subsection{Дисперсия внутренней энергии}

\begin{equation}
    \boxed{(\Delta U)^2 = kT \left[\left(\frac{\partial U}{\partial T}\right)_V^2 \left(\frac{\partial T}{\partial S}\right)_{V,N} - \left(\frac{\partial p}{\partial V}\right)_{T,N} \left(\frac{\partial U}{\partial V}\right)_T^2\right]}
    \tag{12.16}
\end{equation}

%═══════════════════════════════════════════════════════════════════
\section{Полезные соотношения из примеров}
%═══════════════════════════════════════════════════════════════════

\subsection{Дисперсия объёма идеального газа}

Для $N$ молекул под поршнем при давлении $p$ и температуре $T$:
\begin{equation}
    \boxed{(\Delta V)^2 = \frac{V^2}{N} = \frac{N(kT)^2}{p^2}}
\end{equation}

\subsection{Дисперсия числа частиц}

Для системы с переменным числом частиц:
\begin{equation}
    \boxed{(\Delta N)^2 = kT \left(\frac{\partial N}{\partial \mu}\right)_{V,T}}
\end{equation}

\subsection{Дисперсия энергии идеального газа}

Для одноатомного идеального газа из $N$ частиц:
\begin{equation}
    \boxed{(\Delta U)^2 = \frac{3N(kT)^2}{2}}
\end{equation}

\subsection{Корреляция числа частиц и энергии для идеального газа}

Для идеального газа с $(\Delta N)^2 = \bar{N}$ и $U = 3kTN/2$:
\begin{equation}
    \boxed{\overline{\Delta N \Delta H} = \frac{3kT\bar{N}}{2}}
\end{equation}

\subsection{Относительные флуктуации}

Для макроскопических систем ($N \sim 10^{23}$):
\begin{equation}
    \delta_B = \frac{\sigma_B}{\bar{B}} \sim \frac{1}{\sqrt{N}} \sim 10^{-12}
\end{equation}

Флуктуации пренебрежимо малы для макроскопических величин.

\end{document}
