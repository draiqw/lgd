\documentclass[12pt,a4paper]{article}
\usepackage[utf8]{inputenc}
\usepackage[T2A]{fontenc}
\usepackage[russian]{babel}
\usepackage{amsmath,amssymb,amsfonts}
\usepackage{geometry}
\geometry{margin=2cm}

\title{\S 7. Распределения Гиббса}
\author{Конспект формул}
\date{}

\begin{document}
\maketitle

%═══════════════════════════════════════════════════════════════════
\section{Краткая теория}
%═══════════════════════════════════════════════════════════════════

\subsection{Микросостояние системы}

Состояние системы из $N$ частиц определяется координатами и импульсами всех частиц: $X = (x_1, x_2, \ldots, x_N) = (\vec{r}_1, \vec{r}_2, \ldots, \vec{r}_N, \vec{p}_1, \vec{p}_2, \ldots, \vec{p}_N)$, где $x_i = (\vec{r}_i, \vec{p}_i)$.

\subsection{Функция Гамильтона}

Эволюция системы описывается уравнениями Гамильтона с гамильтонианом, включающим кинетическую энергию, внешний потенциал $U_0$ и межчастичное взаимодействие $\Phi$.

\subsection{Типы распределений}

\begin{itemize}
    \item \textbf{Микроканоническое} --- для изолированной системы с фиксированной энергией $E$
    \item \textbf{Каноническое} --- для системы в термостате с фиксированной температурой $T$
    \item \textbf{Большое каноническое} --- для системы с переменным числом частиц (фиксирован химический потенциал $\mu$)
\end{itemize}

\subsection{Энтропия и термодинамика}

Энтропия выражается через функцию распределения и достигает максимума в состоянии равновесия. Это даёт статистическое обоснование второго начала термодинамики.

%═══════════════════════════════════════════════════════════════════
\section{Основные формулы}
%═══════════════════════════════════════════════════════════════════

\subsection{Уравнения Гамильтона}

\begin{equation}
    \boxed{\frac{d\vec{r}_i}{dt} = \frac{\partial H(X)}{\partial \vec{p}_i}, \qquad \frac{d\vec{p}_i}{dt} = -\frac{\partial H(X)}{\partial \vec{r}_i}}
    \tag{7.1}
\end{equation}

%───────────────────────────────────────────────────────────────────
\subsection{Функция Гамильтона}

\begin{equation}
    \boxed{H(X) = \sum_{i=1}^{N} \frac{\vec{p}_i^2}{2m} + U(\vec{r}_1, \ldots, \vec{r}_N) = \sum_{i=1}^{N} \left(\frac{\vec{p}_i^2}{2m} + U_0(\vec{r}_i)\right) + \sum_{i<j} \Phi(|\vec{r}_i - \vec{r}_j|)}
    \tag{7.2}
\end{equation}

где $m$ --- масса частицы, $U_0$ --- потенциал внешнего поля, $\Phi$ --- потенциал межчастичного взаимодействия.

%───────────────────────────────────────────────────────────────────
\subsection{Микроканоническое распределение Гиббса}

Для изолированной системы с энергией $E$:
\begin{equation}
    \boxed{w_N(X, a, E) = \frac{\delta(H(X, a) - E)}{\tilde{\Omega}(a, E)}}
    \tag{7.4}
\end{equation}

\textbf{Статистический вес:}
\begin{equation}
    \tilde{\Omega}(a, E) = \int \delta(H(X, a) - E)\,dX
\end{equation}

%───────────────────────────────────────────────────────────────────
\subsection{Каноническое распределение Гиббса (классическое)}

Для системы в термостате с температурой $T$:
\begin{equation}
    \boxed{w_N(X, a, T) = \frac{\exp(-H(X, a)/kT)}{Z(a, T)}}
    \tag{7.5}
\end{equation}

\textbf{Статистический интеграл:}
\begin{equation}
    \boxed{Z(a, T) = \int \exp(-H(X, a)/kT)\,dX}
\end{equation}

%───────────────────────────────────────────────────────────────────
\subsection{Факторизация канонического распределения}

Если $H(X) = \sum_{\alpha=1}^{m} H_\alpha(X_\alpha)$, то:
\begin{equation}
    \boxed{w_N(X) = \prod_{\alpha=1}^{m} \frac{\exp(-H_\alpha(X_\alpha)/kT)}{Z_\alpha(a, T, N_\alpha)}}
    \tag{7.6}
\end{equation}

%───────────────────────────────────────────────────────────────────
\subsection{Квантовое микроканоническое распределение}

\begin{equation}
    \boxed{P_n(N, a, E) = \frac{\delta_{E, E_n}}{\Omega_1(a, N, E)}}
    \tag{7.7}
\end{equation}

где $\Omega_1 = \sum_n \delta_{E, E_n}$ --- число микросостояний с энергией $E$.

%───────────────────────────────────────────────────────────────────
\subsection{Квантовое каноническое распределение}

\begin{equation}
    \boxed{P_n(N, a, T) = \frac{\exp(-E_n/kT)}{Z_1(a, N, E)} = \exp\left(\frac{F - E_n}{kT}\right)}
    \tag{7.8}
\end{equation}

\textbf{Статистическая сумма:}
\begin{equation}
    \boxed{Z_1 = \sum_n \exp(-E_n/kT)}
    \tag{7.9}
\end{equation}

\textbf{Связь со свободной энергией:}
\begin{equation}
    \boxed{F = -kT \ln Z_1(a, N, E)}
    \tag{7.10}
\end{equation}

%───────────────────────────────────────────────────────────────────
\subsection{Переход от квантовой суммы к классическому интегралу}

\begin{equation}
    \boxed{Z_1 = \sum_n \exp(-E_n/kT) \to \vartheta(N) Z(a, T)}
    \tag{7.11}
\end{equation}

где $\vartheta(N) = [(2\pi\hbar)^{3N} N!]^{-1}$, $\hbar = 1{,}0546 \cdot 10^{-34}$ Дж$\cdot$с.

%───────────────────────────────────────────────────────────────────
\subsection{Большое каноническое распределение}

Для системы с переменным числом частиц:
\begin{equation}
    \boxed{P_{n,N}(a, T, \mu) = \frac{\exp[(\mu N - E_{n,N})/kT]}{Z_2(a, T, \mu)}}
    \tag{7.12}
\end{equation}

\textbf{Большая статистическая сумма:}
\begin{equation}
    \boxed{Z_2 = \sum_{N,n} \exp[(\mu N - E_{n,N})/kT]}
    \tag{7.13}
\end{equation}

\textbf{Связь с большим потенциалом:}
\begin{equation}
    \boxed{\Omega = -kT \ln Z_2 = F - \mu N}
    \tag{7.14}
\end{equation}

%───────────────────────────────────────────────────────────────────
\subsection{Среднее число частиц}

\begin{equation}
    \boxed{N(a, T, \mu) = -\left(\frac{\partial \Omega}{\partial \mu}\right)_{a,T}}
    \tag{7.16}
\end{equation}

%───────────────────────────────────────────────────────────────────
\subsection{Энтропия через функцию распределения}

\textbf{Квантовый случай:}
\begin{equation}
    \boxed{S(a, T, \mu) = -k \sum_{n,N} P_{n,N} \ln P_{n,N}}
    \tag{7.19}
\end{equation}

\textbf{Классический случай:}
\begin{equation}
    \boxed{S(a, T, \mu) = -k \sum_N \vartheta(N) \int w_N \ln(w_N)\,dX = -k \ln w_N}
    \tag{7.20}
\end{equation}

%───────────────────────────────────────────────────────────────────
\subsection{Теорема о равномерном распределении энергии}

\begin{equation}
    \boxed{\bar{H} = \left(\frac{s_1}{2} + \frac{s_2}{\eta}\right)kT}
    \tag{7.21}
\end{equation}

где $s_1$ --- полное число степеней свободы, $s_2$ --- число колебательных степеней свободы.

\textbf{Для потенциала} $U(\lambda\vec{r}_1, \ldots, \lambda\vec{r}_N) = \lambda^\eta U(\vec{r}_1, \ldots, \vec{r}_N)$.

\textbf{Следствие:} На каждую поступательную или вращательную степень свободы приходится энергия $kT/2$, на колебательную --- $kT$.

%═══════════════════════════════════════════════════════════════════
\section{Полезные соотношения из примеров}
%═══════════════════════════════════════════════════════════════════

\subsection{Гауссов интеграл}

\begin{equation}
    \boxed{\int_{-\infty}^{\infty} u^{2n} \exp(-\alpha u^2)\,du = \frac{(2n-1)!!}{(2\alpha)^n} \sqrt{\frac{\pi}{\alpha}}}
    \tag{7.22}
\end{equation}

где $(2n-1)!! = 1 \cdot 3 \cdot 5 \cdots (2n-1)$ при $n \geq 1$, и $(-1)!! = 1$.

%───────────────────────────────────────────────────────────────────
\subsection{Статистический интеграл идеального газа}

Для $N$ частиц в объёме $V$:
\begin{equation}
    \boxed{Z(a, T) = V^N (2\pi mkT)^{3N/2}}
\end{equation}

%───────────────────────────────────────────────────────────────────
\subsection{Свободная энергия идеального газа}

\begin{equation}
    \boxed{F = -kTN\left\{1 + \ln\left[\frac{V(2\pi mkT)^{3/2}}{N(2\pi\hbar)^3}\right]\right\}}
    \tag{7.32}
\end{equation}

%───────────────────────────────────────────────────────────────────
\subsection{Уравнение состояния}

\begin{equation}
    \boxed{p = -\left(\frac{\partial F}{\partial V}\right)_T = \frac{kTN}{V} \quad \Rightarrow \quad pV = \nu RT}
\end{equation}

%───────────────────────────────────────────────────────────────────
\subsection{Внутренняя энергия идеального газа}

\begin{equation}
    \boxed{U = F - T\left(\frac{\partial F}{\partial T}\right)_V = \frac{3kTN}{2} = \frac{3\nu RT}{2}}
\end{equation}

%───────────────────────────────────────────────────────────────────
\subsection{Энтропия идеального газа (формула Сакура--Тетроде)}

\begin{equation}
    \boxed{S = -\left(\frac{\partial F}{\partial T}\right)_{V,N} = kN\left\{\frac{5}{2} + \ln\left[\frac{(2\pi mkT)^{3/2}V}{(2\pi\hbar)^3 N}\right]\right\}}
    \tag{7.33}
\end{equation}

%───────────────────────────────────────────────────────────────────
\subsection{Химический потенциал идеального газа}

\begin{equation}
    \boxed{\mu = kT \ln\left[\frac{N(2\pi\hbar)^3}{V(2\pi mkT)^{3/2}}\right]}
\end{equation}

%───────────────────────────────────────────────────────────────────
\subsection{Теплоёмкость газа в поле тяжести}

\begin{equation}
    \boxed{\tilde{C}_V = \frac{5kN}{2}}
\end{equation}

(больше, чем $3kN/2$ для свободного газа, из-за потенциальной энергии)

%───────────────────────────────────────────────────────────────────
\subsection{Ультрарелятивистский газ ($E = cp$)}

\textbf{Теплоёмкость:}
\begin{equation}
    \boxed{\tilde{C}_V = 3Nk = 3\nu R}
\end{equation}
(в 2 раза больше нерелятивистского)

\textbf{Уравнение состояния:} $pV = \nu RT$ (такое же, как у нерелятивистского газа)

%───────────────────────────────────────────────────────────────────
\subsection{Внутренняя энергия двухатомного газа}

Для молекул с поступательными, вращательными и колебательными степенями свободы:
\begin{equation}
    \boxed{U = \frac{7RT}{2}} \quad (s_1 = 6, \, s_2 = 1, \, \eta = 2)
\end{equation}

\end{document}
