\documentclass[12pt,a4paper]{article}
\usepackage[utf8]{inputenc}
\usepackage[T2A]{fontenc}
\usepackage[russian]{babel}
\usepackage{amsmath,amssymb,amsfonts}
\usepackage{geometry}
\geometry{margin=2cm}

\title{\S 9. Распределение Больцмана}
\author{Конспект формул}
\date{}

\begin{document}
\maketitle

%═══════════════════════════════════════════════════════════════════
\section{Краткая теория}
%═══════════════════════════════════════════════════════════════════

\subsection{Суть распределения Больцмана}

Распределение Больцмана описывает распределение молекул идеального газа по координатам в присутствии внешнего потенциального поля $U_0(\vec{r})$. Является следствием канонического распределения Гиббса при пренебрежении межмолекулярным взаимодействием.

\subsection{Применения}

\begin{itemize}
    \item Распределение молекул в поле тяжести (атмосфера)
    \item Центрифугирование и разделение изотопов
    \item Седиментация взвешенных частиц
\end{itemize}

%═══════════════════════════════════════════════════════════════════
\section{Основные формулы}
%═══════════════════════════════════════════════════════════════════

\subsection{Плотность распределения вероятностей по координатам}

\begin{equation}
    \boxed{w^{(3)}(\vec{r}) = \frac{\exp(-U_0(\vec{r})/kT)}{\displaystyle\int_V \exp(-U_0(\vec{r})/kT)\,d\vec{r}}}
    \tag{9.1}
\end{equation}

где $U_0(\vec{r})$ --- потенциальная энергия молекулы во внешнем поле, $V$ --- объём системы.

%───────────────────────────────────────────────────────────────────
\subsection{Связь с концентрацией}

\begin{equation}
    \boxed{n(\vec{r}) = N w^{(3)}(\vec{r})}
    \tag{9.2}
\end{equation}

где $N$ --- полное число молекул.

%───────────────────────────────────────────────────────────────────
\subsection{Распределение в поле тяжести}

Для потенциала $U_0(z) = mgz$:
\begin{equation}
    \boxed{w(z) = \frac{mg}{kT} \exp\left(-\frac{mgz}{kT}\right)}
    \tag{9.3}
\end{equation}

\textbf{Высота центра масс газа:}
\begin{equation}
    \boxed{z_c = \bar{z} = \frac{kT}{mg}}
    \tag{9.4}
\end{equation}

%───────────────────────────────────────────────────────────────────
\subsection{Барометрическая формула}

\textbf{Концентрация:}
\begin{equation}
    \boxed{n(z) = n(0) \exp\left(-\frac{m_0 gz}{kT}\right)}
    \tag{9.5}
\end{equation}

\textbf{Давление:}
\begin{equation}
    \boxed{p(z) = p(0) \exp\left(-\frac{m_0 gz}{kT}\right)}
    \tag{9.6}
\end{equation}

%───────────────────────────────────────────────────────────────────
\subsection{Газ в цилиндрическом сосуде высотой $H$}

\begin{equation}
    \boxed{w(z) = \frac{mg}{kT} \cdot \frac{\exp(-mgz/kT)}{1 - \exp(-mgH/kT)}}
    \tag{9.7}
\end{equation}

\textbf{Средняя потенциальная энергия:}
\begin{equation}
    \bar{U}_0 = kT - \frac{mgH}{\exp(mgH/kT) - 1}
\end{equation}

\textbf{Положение центра масс:}
\begin{equation}
    \boxed{z_c = \frac{kT}{mg} - \frac{H}{\exp(mgH/kT) - 1}}
    \tag{9.8}
\end{equation}

\textbf{Предельные случаи:}
\begin{itemize}
    \item $mgH \ll kT$: $\bar{U}_0 \approx mgH/2$, $z_c \approx H/2$
    \item $mgH \gg kT$: $\bar{U}_0 \approx kT$, $z_c \approx kT/mg$
\end{itemize}

%═══════════════════════════════════════════════════════════════════
\section{Полезные соотношения из примеров}
%═══════════════════════════════════════════════════════════════════

\subsection{Флуктуации потенциальной энергии}

Для газа в поле тяжести:
\begin{equation}
    \boxed{\bar{U}_0 = kT, \qquad \sigma^2(U_0) = (kT)^2, \qquad \delta(U_0) = 1}
\end{equation}

(относительная флуктуация равна единице!)

%───────────────────────────────────────────────────────────────────
\subsection{Центрифугирование}

\textbf{Потенциальная энергия} во вращающейся системе отсчёта:
\begin{equation}
    U_0 = -\frac{m\omega^2 r^2}{2}
\end{equation}

\textbf{Распределение концентрации:}
\begin{equation}
    \boxed{\frac{n_i(r)}{n_i(0)} = \exp\left(\frac{m_i \omega^2 r^2}{2kT}\right)}
    \tag{9.10}
\end{equation}

\textbf{Коэффициент разделения изотопов:}
\begin{equation}
    \boxed{q = \frac{n_1(R)/n_1(0)}{n_2(R)/n_2(0)} = \exp\left[\frac{(m_1 - m_2)\omega^2 R^2}{2kT}\right]}
\end{equation}

Коэффициент разделения увеличивается при:
\begin{itemize}
    \item уменьшении температуры
    \item увеличении разности масс $|m_1 - m_2|$
    \item увеличении $\omega$ и $R$
\end{itemize}

%───────────────────────────────────────────────────────────────────
\subsection{Толщина слоя с заданным изменением концентрации}

При изменении концентрации на $\Delta n/n$:
\begin{equation}
    \boxed{\Delta z \approx \frac{kT}{mg} \cdot \frac{\Delta n}{n}}
\end{equation}

%───────────────────────────────────────────────────────────────────
\subsection{Центр масс смеси газов в поле тяжести}

Для смеси из $N_1$ частиц массой $m_1$ и $N_2$ частиц массой $m_2$:
\begin{equation}
    \boxed{z_c = \frac{(N_1 + N_2)kT}{g(N_1 m_1 + N_2 m_2)}}
\end{equation}

%───────────────────────────────────────────────────────────────────
\subsection{Доля молекул выше определённой высоты}

Для полубесконечного столба газа, доля молекул на высоте $z > z_0$:
\begin{equation}
    \int_{z_0}^{\infty} w(z)\,dz = \exp\left(-\frac{mgz_0}{kT}\right)
\end{equation}

\textbf{Пример:} на высоте $z > 4z_c = 4kT/mg$ находится менее $e^{-4} \approx 1{,}8\%$ молекул.

\end{document}
